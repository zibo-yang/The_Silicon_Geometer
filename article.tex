\documentclass[12pt]{scrartcl}

\usepackage[utf8]{inputenc}
\usepackage[T1]{fontenc}
\usepackage{babel}

\usepackage{amscd, amssymb, mathtools, amsthm, amsmath, mathrsfs}
\usepackage{enumerate}

% Palatino for main text and math
\usepackage[osf,sc]{mathpazo}

% Helvetica for sans serif
% (scaled to match size of Palatino)
\usepackage[scaled=0.90]{helvet}

% Bera Mono for monospaced
% (scaled to match size of Palatino)
\usepackage[scaled=0.85]{beramono}

\usepackage{hyperref}

% To write appendixes
\usepackage[page]{appendix}


\renewcommand\thesubsection {\arabic{subsection}}
%
%
\newenvironment{eq}{\begin{equation}}{\end{equation}}
%
\newenvironment{remarks}{{\bf Remarks}:\begin{enumerate}}{\end{enumerate}}
\newenvironment{examples}{{\bf Examples}:\begin{enumerate}}{\end{enumerate}}  
%
\newtheorem{proposition}{Proposition}[section]
\newtheorem{lemma}[proposition]{Lemma}
\newtheorem{definition}[proposition]{Definition}
\newtheorem{theorem}[proposition]{Theorem}
\newtheorem{cor}[proposition]{Corollary}
\newtheorem{conjecture}{Conjecture}
\newtheorem{pretheorem}[proposition]{Pretheorem}
\newtheorem{hypothesis}[proposition]{Hypothesis}
\newtheorem{example}[proposition]{Example}
\newtheorem*{rem}{Remark}
\newtheorem{remark}[proposition]{Remark}
\newtheorem{ex}[proposition]{Exercise}
\newtheorem*{hint}{Hint}
\newtheorem{cond}[proposition]{Conditions}
\newtheorem{cons}[proposition]{Construction}
%
\newtheorem{problem}[proposition]{Problem}
\newtheorem{construction}[proposition]{Construction}
%
\newtheorem*{nota}{Notation}
\newtheorem{notation}{Notation}
\newtheorem*{axiom}{Axiom}

% For diagrams:
\usepackage{tikz-cd}
\usetikzlibrary{cd}

% To avoid automatic identation of paragraphs:
\usepackage[parfill]{parskip}

% To create an index:
\usepackage{makeidx}

% To make the index and bibliography appear in the table of contents:
\usepackage[nottoc]{tocbibind}

% notations:
\def\spec{\text{Spec}\,R}

\makeindex

\usepackage{isabelle,isabellesym}

\begin{document}

\title{Simple Type Theory is not too Simple}
\subtitle{Grothendieck's Schemes in the Proof Assistant Isabelle}
\author{Anthony Bordg\thanks{apdb3@cam.ac.uk} \and Lawrence Paulson \and Wenda Li}
\date\today 
\maketitle

\begin{abstract}
 	We report on a formalization of Grothendieck's schemes in the proof assistant Isabelle and we discuss in the process the design choices made. Schemes being sophisticated mathematical objects in algebraic geometry introduced by the mathematician Alexander Grothendieck in 1960, this experiment shows that the simple type theory implemented in Isabelle can easily handle such sophisticated objects despite doubts raised about Isabelle capability in that direction.   
\end{abstract}

\section{Motivation and Problems}
\label{sec:intro}

Proof assistants have made impressive progress on the formalisation of algebra \cite{oddorderthm}. In 2003, Laurent Chicli, as the topic of his PhD thesis \cite{chiclithesis}, formalized sheaves and affine schemes in the proof assistant Coq \cite{coqrefmanual}, which is based on a powerful dependent type theory known as the \emph{Calculus of Inductive Constructions} \cite{coquand1986calculus, coquand1988inductively}. More recently, in 2019 Grothendieck's schemes \cite{GrothendieckEGAI} have been formalized by a group of mainly six mathematicians led by Kevin Buzzard \cite{schemesinLean} using the brave new Lean theorem prover, which is based on a similar dependent type theory \cite{de2015lean}. \\
For the reader unfamiliar with schemes I will quote Kevin Buzzard.
\begin{quote}
	Schemes are the fundamental objects of study in algebraic geometry. They were discovered (in their current form) by Grothendieck in the late 1950s and early 1960s and they revolutionised the theory of algebraic geometry.\footnote{\url{https://github.com/leanprover-community/mathlib/issues/26}} 
\end{quote}
and
\begin{quote}
	A scheme is a mathematical object whose definition and basic properties are usually taught at MSc or early PhD level in a typical mathematics department.
\end{quote}	
Kevin Buzzard in 2017 proposed the formalization of schemes 
\begin{quote}
	as basically a challenge to see if Lean can handle such a complex definition.\footnote{\url{https://github.com/leanprover-community/mathlib/issues/26}} 
\end{quote}
We should note that the Isabelle proof assistant is based on a much more minimalist type theory, known as simple type theory. As a consequence, many users of dependent type theories see Isabelle's type theory as a far less expressive system for formalizing advanced mathematics.  These doubts have been vocally expressed by Kevin Buzzard on his blog:
\begin{quote}
	What can Isabelle/HOL actually do before it breaks? Nobody knows. [\dots] But do you people want to attract “working mathematicians”? Then where are the schemes? Can your system even do schemes? I don’t know. Does anyone know? If it cannot then this would be very valuable to know because it will help mathematician early adopters to make an informed decision about which system to use.\footnote{\url{https://xenaproject.wordpress.com/2020/02/09/where-is-the-fashionable-mathematics/}}
\end{quote}
In this article we present a concise and elegant formalization of schemes in Isabelle/HOL based on state-of-the-art locales. Our presentation follows the classic graduate textbook of Hartshorne \cite{hartshorne}. 			


\section{Isabelle/HOL, Simple Type Theory and Locales}
\label{sec:isabelle}

Isabelle/HOL~\cite{isa-tutorial} is a proof assistant for higher-order logic---Church's simple type theory~\cite{church40}---and is therefore based on the typed $\lambda$-calculus with boolean and function types. A boolean-valued function is a predicate, which yields a simple typed set theory that is expressive enough to express abstract mathematics, as we demonstrate below. 

In Church's original conception, the sole purpose of types is to prevent inconsistency, all collections being represented as sets. People today, influenced by programming languages, prefer to give types a more dominant role. They expect to have numeric types like \isa{int}, \isa{nat}, \isa{real}, \isa{complex} and for types to play a major role in logical reasoning. For example, an expression like \isa{x+y} should be interpreted with respect to the types of \isa{x} and~\isa{y}.
Isabelle/HOL's \emph{type classes} \cite{wenzel-type} support this sort of type-based disambiguation in a strong sense, so that \isa{x+y} refers to the correct instance of addition.
Type classes work for such trivialities as the basic laws for addition and multiplication (with the help of type classes for groups and rings) and for more advanced topological properties~\cite{hoelzl-filters}. A new type, once proved to be a topological space, instantly inherits such concepts as limits and continuity along with all theorems proved about them.

The drawback of type classes is simply that they refer to types. Mathematical constructions are generally too complex to formalise as types. Even the carrier of a group can be a complex object. Moving to a stronger type system, as done in Coq or Lean, brings issues of its own. Our solution is to define what we need using the naive set theory that comes with higher-order logic, through the mechanism of locales.%
\footnote{Not to be confused with the locales of point-free topology!}

A \emph{locale}~\cite{ballarin-locales-module} represents an Isabelle/HOL proof context. It can take parameters and instances of other locales. It can declare components constrained by assumptions. Thus they correspond directly to the mathematical practice of defining a monoid, say, as a tuple $(M,\cdot ,1)$ satisfying the obvious properties. 
A locale can inherit multiple instances, as when we define a monoid homomorphism with respect to two monoids.

Logically, a locale is nothing but a predicate; their power comes from Isabelle's mechanisms for creating, managing and using locale hierarchies.
If we work in a particular locale, we can refer to its components and assumptions. But we can also prove membership of a locale, which means to exhibit particular constructions---say a purported monoid carrier along with its multiplication and identity---and prove that they satisfy the locale's assumptions. \emph{Sublocale} declarations provide a means of proving inclusions between apparently unrelated locales.
We found that locales handled the tangle of definitions building up to schemes straightforwardly.

However, the right way to formalise algebra is not obvious.
It is well-known in the Isabelle community that the main algebra library needs to be rebuilt from scratch. In fact, there are many libraries for algebra in Isabelle: \textit{HOL-Algebra}\footnote{\url{https://isabelle.in.tum.de/dist/library/HOL/HOL-Algebra/index.html}} from the previous millennium (1999), the more recent \textit{HOL-Computational\_Algebra}\footnote{\url{https://isabelle.in.tum.de/dist/library/HOL/HOL-Computational_Algebra/index.html}}, and moreover some entries in the Archive of Formal Proofs\footnote{https://www.isa-afp.org/} like \textit{Groups, Rings and Modules} \cite{Group-Ring-Module-AFP} and \textit{Vector Spaces} \cite{VectorSpace-AFP} among others. As a result, there is a lot of overlapping material, which makes the status of algebra confusing and unclear for the newcomer in the Isabelle world. And while these formalizations constitute an impressive body of work, parts of it are deprecated code not using the Isar language or locales. 

Even using locales, we can still go wrong. The root of the problems in HOL-Algebra is the desire to refer to a structure such as a group with its components using a single variable, as a record data structure. The new (at the time) extensible records seemed perfect for the task.
But they led to some peculiarities: notably, the locale \emph{abelian\_group} (in \textit{Theory Ring}) which presents the odd twist of requiring a ring structure. The lack of multiple inheritance for records seems to have required the awkward use of the ring record for abelian groups. However, Clemens Ballarin recently conducted a successful experiment \cite{ballarin-exploring} showing that  locales, without records, allow for a smooth handling of basic algebraic structures in Isabelle. 


\section{Schemes in Isabelle/HOL}
\label{sec:schemes}

For the remainder of this text fix a commutative ring $R$. \\
The definition of an ideal of $R$ was introduced in Isabelle  by Ballarin~\cite{ballarin-exploring}.

\begin{definition}[ideal]
	An ideal $\mathfrak{a}$ of $R$ is a subset of $R$ which is an additive subgroup such that $R\mathfrak{a} = \mathfrak{a}$.
\end{definition}

% Snippet of code to be inserted here for `locale ideal` in Jacobson_Basic_Algebra.Ring_Theory.thy

\begin{isabelle}
\isacommand{locale}\ ideal\ =\ subgroup\_of\_additive\_group\_of\_ring\ +\isanewline
\ \ \isakeyword{assumes}\ ideal:\ "\isasymlbrakk \ a\ \isasymin \ R;\ b\ \isasymin \ I\ \isasymrbrakk \ \isasymLongrightarrow \ a\ \isasymcdot \ b\ \isasymin \ I"\ \isanewline
\ \ \ \ \ \ \ \ \ \ \ \ \ \ \ \ \ "\isasymlbrakk \ a\ \isasymin \ R;\ b\ \isasymin \ I\ \isasymrbrakk \ \isasymLongrightarrow \ b\ \isasymcdot \ a\ \isasymin \ I"
\end{isabelle}

\begin{rem}
	Note that in the code above the ideal is denoted $I$ while the beloved gothic letters are standard in mathematics for denoting ideals. In our code we use both depending on the situation.
\end{rem}	

One easily checks that $\lbrace 0 \rbrace$ and $R$ are ideals of $R$. In our formalization this translates as follows.

% snippet of code to be inserted for `lemma ideal_R_R` and `lemma ideal_0_R` of Comm_Ring_Theory.thy

\begin{isabelle}
\isacommand{lemma}\ (\isakeyword{in}\ comm\_ring)\ ideal\_R\_R:\ "ideal\ R\ R\ (+)\ (\isasymcdot )\ \isasymzero \ \isasymone "
\end{isabelle}

\begin{isabelle}
\isacommand{lemma}\ (\isakeyword{in}\ comm\_ring)\ ideal\_0\_R:\ "ideal\ \{\isasymzero \}\ R\ (+)\ (\isasymcdot )\ \isasymzero \ \isasymone "
\end{isabelle}

Given $\mathfrak{a}$ and $\mathfrak{b}$ two ideals of $R$, one defines $\mathfrak{a} \mathfrak{b}$ to be the ideal whose underlying set is 
\[
	\lbrace a_1 b_1 + \dots + a_n b_n \mid a_i \in \mathfrak{a}, b_i \in \mathfrak{b}, n \in \mathbb{N} \rbrace.
\]

Actually, $\mathfrak{a} \mathfrak{b}$ is the smallest ideal generated by the set 
	\[
	\lbrace a b \mid a \in \mathfrak{a}, b \in \mathfrak{b} \rbrace \, .
	\]
	
% snippet of code to be inserted for `ideal_gen_by_prod` of Comm_Ring_Theory.thy	

\begin{isabelle}
\isacommand{definition}\ (\isakeyword{in}\ comm\_ring)\ ideal\_gen\_by\_prod\ \isanewline
\ \ \ \ \ \ \ \ \ \ \ \ \ \ \ \ \ \ \ \ \ \ \ \ \ \ \ \ \ \ ::\ "'a\ set\ \isasymRightarrow \ 'a\ set\ \isasymRightarrow \ 'a\ set"\isanewline
\ \ \isakeyword{where}\ "ideal\_gen\_by\_prod\ \isasymaa \ \isasymbb \ \isasymequiv \ additive.subgroup\_generated\ \isanewline
\ \ \ \ \ \ \ \ \ \ \ \ \ \ \ \ \ \ \ \ \ \ \ \ \{x.\ \isasymexists a\ b.\ x\ =\ a\ \isasymcdot \ b\ \isasymand \ a\ \isasymin \ \isasymaa \ \isasymand \ b\ \isasymin \ \isasymbb \}"
\end{isabelle}

One easily checks $\mathfrak{a} \mathfrak{b}$ is an ideal.

% snippet of code to be inserted for `lemma ideal_subgroup_generated` of Comm_Ring_Theory.thy

\begin{isabelle}
\isacommand{lemma}\ (\isakeyword{in}\ comm\_ring)\ ideal\_subgroup\_generated:\isanewline
\ \ \isakeyword{assumes}\ "ideal\ \isasymaa \ R\ (+)\ (\isasymcdot )\ \isasymzero \ \isasymone "\ \isakeyword{and}\ "ideal\ \isasymbb \ R\ (+)\ (\isasymcdot )\ \isasymzero \ \isasymone "\isanewline
\ \ \isakeyword{shows}\ "ideal\ (ideal\_gen\_by\_prod\ \isasymaa \ \isasymbb )\ R\ (+)\ (\isasymcdot )\ \isasymzero \ \isasymone "
\end{isabelle}

This ideal is in fact the intersection of all ideals of $R$ containing the set
	\[
	\lbrace a b \mid a \in \mathfrak{a}, b \in \mathfrak{b} \rbrace .
	\]
	
% snippet of code to be inserted for `lemma ideal_gen_by_prod_is_Inter`	of Comm_Ring_Theory.thy
\begin{isabelle}
\isacommand{lemma}\ (\isakeyword{in}\ comm\_ring)\ ideal\_gen\_by\_prod\_is\_inter:\isanewline
\ \ \isakeyword{assumes}\ "ideal\ \isasymaa \ R\ (+)\ (\isasymcdot )\ \isasymzero \ \isasymone "\ \isakeyword{and}\ "ideal\ \isasymbb \ R\ (+)\ (\isasymcdot )\ \isasymzero \ \isasymone "\isanewline
\ \ \isakeyword{shows}\ "ideal\_gen\_by\_prod\ \isasymaa \ \isasymbb \ =\ \isasymInter \ \{I.\ ideal\ I\ R\ (+)\ (\isasymcdot )\ \isasymzero \ \isasymone \ \isanewline
\ \ \ \ \ \ \ \ \ \ \ \ \ \ \ \ \ \ \ \ \ \ \ \ \ \ \isasymand \ \{a\ \isasymcdot \ b\ |a\ b.\ a\ \isasymin \ \isasymaa \ \isasymand \ b\ \isasymin \ \isasymbb \}\ \isasymsubseteq \ I\}"
\end{isabelle}

Given $\lbrace \mathfrak{a}_i \rbrace$ a family of ideals of $R$ indexed by an arbitrary set $I$, one defines the set $\displaystyle \sum_{i \in I} \mathfrak{a}_i$ to be the set 
	\[
	\lbrace a_1 + \dots + a_n \mid n \in \mathbb{N}, a_i \in \mathfrak{a}_i \rbrace .
	\]

The set $\displaystyle \sum_{i \in I} \mathfrak{a}_i$ is again an ideal of $R$.	

The so-called prime ideals are an important class of ideals. A prime ideal $\mathfrak{p}$ of $R$ is an ideal $\mathfrak{p} \neq R$ such that $R/\mathfrak{p}$ is an entire ring. However, in a formal context we prefer the following equivalent definition which is more effective.

\begin{definition}[prime ideal]
	A prime ideal is an ideal $\mathfrak{p} \neq R$ such that $x y \in \mathfrak{p}$ implies $x \in \mathfrak{p}$ or $y \in \mathfrak{p}$ for every elements $x$ and $y$ in $R$.
\end{definition}

% snippet of code to be inserted `locale pr_ideal` of Comm_Ring_Theory.thy
\begin{isabelle}
\isacommand{locale}\ pr\_ideal\ =\ comm:comm\_ring\ R\ "(+)"\ "(\isasymcdot )"\ "\isasymzero "\ "\isasymone "\ \isanewline
\ \ \ \ \ \ \ \ \ \ \ \ \ \ \ \ \ \ \ \ \ \ \ \ +\ ideal\ I\ R\ "(+)"\ "(\isasymcdot )"\ "\isasymzero "\ "\isasymone "\isanewline
\ \ \ \ \isakeyword{for}\ R\ \isakeyword{and}\ I\ \isakeyword{and}\ addition\ (\isakeyword{infixl}\ "+"\ 65)\ \isakeyword{and}\ multiplication\ \isanewline
\ \ \ \ \ \ \ (\isakeyword{infixl}\ "\isasymcdot "\ 70)\ \isakeyword{and}\ zero\ ("\isasymzero ")\ \isakeyword{and}\ unit\ ("\isasymone ")\isanewline
\ \ \ +\ \isakeyword{assumes}\ carrier\_neq:\ "I\ \isasymnoteq \ R"\ \isanewline
\ \ \ \ \ \ \ \ \ \ \isakeyword{and}\ absorbent:\ "\isasymlbrakk x\ \isasymin \ R;\ y\ \isasymin \ R\isasymrbrakk \ \isasymLongrightarrow \ (x\ \isasymcdot \ y\ \isasymin \ I)\ \isanewline
\ \ \ \ \ \ \ \ \ \ \ \ \ \ \ \ \ \ \ \ \ \ \ \ \ \ \ \ \ \ \ \ \ \ \ \ \ \ \ \ \ \ \isasymLongrightarrow \ (x\ \isasymin \ I\ \isasymor \ y\ \isasymin \ I)"
\end{isabelle}

Note that if $\mathfrak{p}$ is a prime ideal, then $1 \notin \mathfrak{p}$.

% snippet of code to be inserted for the statement and proof of `lemma not_1` of Comm_Ring_Theory.thy

\begin{isabelle}
\isacommand{lemma}\ (\isakeyword{in}\ pr\_ideal)\ not\_1:\ "\isasymone \ \isasymnotin \ I"\isanewline
\isacommand{proof}\isanewline
\ \ \isacommand{assume}\ "\isasymone \ \isasymin \ I"\isanewline
\ \ \isacommand{then}\ \isacommand{have}\ "\isasymAnd x.\ \isasymlbrakk \isasymone \ \isasymin \ I;\ x\ \isasymin \ R\isasymrbrakk \ \isasymLongrightarrow \ x\ \isasymin \ I"\isanewline
\ \ \ \ \isacommand{by}\ (metis\ ideal(1)\ comm.multiplicative.right\_unit)\isanewline
\ \ \isacommand{with}\ \isacartoucheopen \isasymone \ \isasymin \ I\isacartoucheclose \ \isacommand{have}\ "I\ =\ R"\isanewline
\ \ \ \ \isacommand{by}\ auto\isanewline
\ \ \isacommand{then}\ \isacommand{show}\ False\isanewline
\ \ \ \ \isacommand{using}\ carrier\_neq\ \isacommand{by}\ blast\isanewline
\isacommand{qed}
\end{isabelle}

Let $\mathfrak{p}$ be a prime ideal and $S$ be the complement of $\mathfrak{p}$ in $R$,  we prove $S$ is a multiplicative submonoid of $R$.

% snippet of code to be inserted for `lemma submonoid_notin` of Comm_Ring_Theory.thy	
\begin{isabelle}
\isacommand{lemma}\ (\isakeyword{in}\ pr\_ideal)\ submonoid\_notin:\isanewline
\ \ \isakeyword{assumes}\ "S\ =\ \{x\ \isasymin \ R.\ x\ \isasymnotin \ I\}"\isanewline
\ \ \isakeyword{shows}\ "submonoid\ S\ R\ (\isasymcdot )\ \isasymone "
\end{isabelle}

If $\mathfrak{a}$ is an ideal of $R$, $V(\mathfrak{a})$ will denote the set of all prime ideals of $R$ which contain $\mathfrak{a}$. 

% snippet of code to be inserted `definition closed_subsets` of Comm_Ring_Theory.thy
\begin{isabelle}
\isacommand{definition}\ (\isakeyword{in}\ comm\_ring)\ closed\_subsets\isanewline
\ \ \ \ \ \ \ \ \ \ \ \ \ ::\ "'a\ set\ \isasymRightarrow \ ('a\ set)\ set"\ ("\isasymV \ \_"\ [900]\ 900)\isanewline
\ \ \isakeyword{where}\ "\isasymV \ \isasymaa \ \isasymequiv \ \{I.\ pr\_ideal\ R\ I\ (+)\ (\isasymcdot )\ \isasymzero \ \isasymone \ \isasymand \ \isasymaa \ \isasymsubseteq \ I\}"
\end{isabelle}

Note that $V(R) = \emptyset$ and $V(\lbrace 0 \rbrace)$ is the set of all prime ideals of $R$.

% snippet of code to be inserted for `lemma closed_subsets_empty` and `lemma closed_subsets_R` of Comm_Ring_Theory.thy
\begin{isabelle}
\isacommand{lemma}\ (\isakeyword{in}\ comm\_ring)\ closed\_subsets\_empty:\ "\isasymV \ \{\}\ =\ Spec"
\end{isabelle}

\begin{isabelle}
\isacommand{lemma}\ (\isakeyword{in}\ comm\_ring)\ closed\_subsets\_R:\ "\isasymV \ R\ =\ \{\}"
\end{isabelle}

Then, one proves in Isabelle the subsets of the form $V(\mathfrak{a})$ as the closed subsets define a topology, the so-called \emph{Zariski topology}\label{zariskitop}, on the set of all primes ideals of $R$. In the rest of this text $\text{Spec}\,R$ will denote the set of all prime ideals of $R$ equipped with its Zariski topology.

% snippets of code to be inserted here `definition is_zariski_open` and `zariski_is_topological_space` of Comm_Ring_Theory.thy
\begin{isabelle}
\isacommand{definition}\ (\isakeyword{in}\ comm\_ring)\ is\_zariski\_open\isanewline
\ \ \ \ ::\ "'a\ set\ set\ \isasymRightarrow \ bool"\ \isakeyword{where}\isanewline
\ \ "is\_zariski\_open\ U\ \isasymequiv \ generated\_topology\ Spec\ \isanewline
\ \ \ \ \ \ \ \ \{U.\ (\isasymexists \isasymaa .\ ideal\ \isasymaa \ R\ (+)\ (\isasymcdot )\ \isasymzero \ \isasymone \ \isasymand \ U\ =\ Spec\ -\ \isasymV \ \isasymaa )\}\ U"
\end{isabelle}

\begin{isabelle}
\isacommand{lemma}\ (\isakeyword{in}\ comm\_ring)\ zariski\_is\_topological\_space:\isanewline
\ \ "topological\_space\ Spec\ is\_zariski\_open"
\end{isabelle}

We now teach Isabelle what \emph{presheaves of rings} are. 

\begin{definition}[presheaf of rings]
	Let $X$ be a topological space. A presheaf $\mathscr{F}$ of rings on $X$ consists of the following data:
	\begin{itemize}
		\item for every open set $U$, a ring $\mathscr{F}(U)$
		\item for every inclusion $V \subseteq U$ of open subsets, a morphism of rings $\rho_{UV}: \mathscr{F}(U) \rightarrow \mathscr{F}(V)$  
	\end{itemize}
satisfying 
	\begin{enumerate}
		\item $\mathscr{F}(\emptyset) = \lbrace 0 \rbrace$
		\item $\rho_{UU}$ is the identity map for every open subset $U$
		\item  If $W \subseteq V \subseteq U$ are three open subsets, then $\rho_{UW} = \rho_{VW} \circ \rho_{UV}$.
	\end{enumerate}
\end{definition}

% snippet of code to be inserted `locale presheaf_of_rings` of Comm_Ring_Theory.thy

\begin{isabelle}
\isacommand{locale}\ presheaf\_of\_rings\ =\ topological\_space\isanewline
\ \ +\ \isakeyword{fixes}\ \isasymFF ::\ "'a\ set\ \isasymRightarrow \ 'b\ set"\isanewline
\ \ \isakeyword{and}\ \isasymrho ::\ "'a\ set\ \isasymRightarrow \ 'a\ set\ \isasymRightarrow \ ('b\ \isasymRightarrow \ 'b)"\ \isakeyword{and}\ b::\ "'b"\isanewline
\ \ \isakeyword{and}\ add\_str::\ "'a\ set\ \isasymRightarrow \ ('b\ \isasymRightarrow \ 'b\ \isasymRightarrow \ 'b)"\ ("+\isactrlbsub \_\isactrlesub ")\isanewline
\ \ \isakeyword{and}\ mult\_str::\ "'a\ set\ \isasymRightarrow \ ('b\ \isasymRightarrow \ 'b\ \isasymRightarrow \ 'b)"\ ("\isasymcdot \isactrlbsub \_\isactrlesub ")\isanewline
\ \ \isakeyword{and}\ zero\_str::\ "'a\ set\ \isasymRightarrow \ 'b"\ ("\isasymzero \isactrlbsub \_\isactrlesub ")\ \isanewline
\ \ \isakeyword{and}\ one\_str::\ "'a\ set\ \isasymRightarrow \ 'b"\ ("\isasymone \isactrlbsub \_\isactrlesub ")\isanewline
\ \ \isakeyword{assumes}\ is\_ring\_morphism:\isanewline
\ \ \ \ "\isasymAnd U\ V.\ is\_open\ U\ \isasymLongrightarrow \ is\_open\ V\ \isasymLongrightarrow \ V\ \isasymsubseteq \ U\ \isanewline
\ \ \ \ \ \ \ \ \ \ \ \ \ \ \ \ \isasymLongrightarrow \ ring\_homomorphism\ (\isasymrho \ U\ V)\isanewline
\ \ \ \ \ \ \ \ \ \ \ \ \ \ \ \ \ \ \ \ \ \ \ \ \ \ (\isasymFF \ U)\ (+\isactrlbsub U\isactrlesub )\ (\isasymcdot \isactrlbsub U\isactrlesub )\ \isasymzero \isactrlbsub U\isactrlesub \ \isasymone \isactrlbsub U\isactrlesub \isanewline
\ \ \ \ \ \ \ \ \ \ \ \ \ \ \ \ \ \ \ \ \ \ \ \ \ \ (\isasymFF \ V)\ (+\isactrlbsub V\isactrlesub )\ (\isasymcdot \isactrlbsub V\isactrlesub )\ \isasymzero \isactrlbsub V\isactrlesub \ \isasymone \isactrlbsub V\isactrlesub "\isanewline
\ \ \isakeyword{and}\ ring\_of\_empty:\ "\isasymFF \ \{\}\ =\ \{b\}"\isanewline
\ \ \isakeyword{and}\ identity\_map\ [simp]:\ "\isasymAnd U.\ is\_open\ U\ \isanewline
\ \ \ \ \ \ \ \ \ \ \ \ \ \ \ \ \ \ \ \ \ \ \isasymLongrightarrow \ (\isasymAnd x.\ x\ \isasymin \ \isasymFF \ U\ \isasymLongrightarrow \ \isasymrho \ U\ U\ x\ =\ x)"\isanewline
\ \ \isakeyword{and}\ assoc\_comp:\ \isanewline
\ \ \ \ "\isasymAnd U\ V\ W\ x.\ \isasymlbrakk is\_open\ U;\ is\_open\ V;\ is\_open\ W;V\ \isasymsubseteq \ U;\ \isanewline
\ \ \ \ \ \ \ \ \ W\ \isasymsubseteq \ V;x\ \isasymin \ (\isasymFF \ U)\isasymrbrakk \ \isasymLongrightarrow \ \isasymrho \ U\ W\ x\ =\ (\isasymrho \ V\ W\ \isasymcirc \ \isasymrho \ U\ V)\ x"
\end{isabelle}

Given the condition \textit{is\_ring\_morphism} within the locale, notice it is possible to prove that the $\mathscr{F}(U)$'s are rings for every open subset $U$ of the underlying topological space.

% snippet of code to be inserted `is_ring_from_is_homomorphism` of Comm_Ring_Theory.thy
\begin{isabelle}
\isacommand{lemma}\ (\isakeyword{in}\ presheaf\_of\_rings)\ is\_ring\_from\_is\_homomorphism:\isanewline
\ \ \isakeyword{fixes}\ U::\ "'a\ set"\isanewline
\ \ \isakeyword{assumes}\ "is\_open\ U"\isanewline
\ \ \isakeyword{shows}\ "ring\ (\isasymFF \ U)\ (+\isactrlbsub U\isactrlesub )\ (\isasymcdot \isactrlbsub U\isactrlesub )\ \isasymzero \isactrlbsub U\isactrlesub \ \isasymone \isactrlbsub U\isactrlesub "
\end{isabelle}

\begin{notation}
	The elements of $\mathscr{F}(U)$ are sometimes called the sections of the presheaf $\mathscr{F}$ and given $s \in \mathscr{F}(U)$, $s\restriction V$ denotes the element $\rho_{UV}(s)$.
\end{notation}

Of course, we have the corresponding notion of morphisms.

\begin{definition}[morphism of presheaves of rings]
	A morphism $\phi: \mathscr{F} \rightarrow \mathscr{F}'$ of sheaves of rings on a topological space $X$ is given by a morphism $\phi_U: \mathscr{F}(U) \rightarrow \mathscr{F}'(U)$ for each open subset $U$ of $X$ such that  the following diagram commutes
	\[
	\begin{tikzcd}
	\mathscr{F}(U) \arrow[r, "\phi_U"] \arrow[d, "\rho_{UV}"] & \mathscr{F}'(U) \arrow[d, "\rho'_{UV}"] \\
	\mathscr{F}(V) \arrow[r, "\phi_V"] & \mathscr{F}'(V) 
	\end{tikzcd}
	\]
	for every inclusion $V \subseteq U$.
\end{definition}

% snippet of code to be inserted `locale morphism_presheaves_of_rings`  of Comm_Ring_Theory.thy
\begin{isabelle}
\isacommand{locale}\ morphism\_presheaves\_of\_rings\ =\isanewline
\ \ \ \ source:\ presheaf\_of\_rings\ X\ is\_open\ \isasymFF \ \isasymrho \ b\ add\_str\isanewline
\ \ \ \ \ \ \ \ \ \ \ \ \ \ \ \ \ \ \ \ \ \ \ \ \ \ \ \ \ \ \ \ mult\_str\ zero\_str\ one\_str\isanewline
\ \ +\ target:\ presheaf\_of\_rings\ X\ is\_open\ \isasymFF '\ \isasymrho '\ b'\ add\_str'\ \isanewline
\ \ \ \ \ \ \ \ \ \ \ \ \ \ \ \ \ \ \ \ \ \ \ \ \ \ \ \ \ mult\_str'\ zero\_str'\ one\_str'\isanewline
\ \ \isakeyword{for}\ X\ \isakeyword{and}\ is\_open\isanewline
\ \ \ \ \isakeyword{and}\ \isasymFF \ \isakeyword{and}\ \isasymrho \ \isakeyword{and}\ b\ \isakeyword{and}\ add\_str\ ("+\isactrlbsub \_\isactrlesub ")\ \isanewline
\ \ \ \ \isakeyword{and}\ mult\_str\ ("\isasymcdot \isactrlbsub \_\isactrlesub ")\ \isakeyword{and}\ zero\_str\ ("\isasymzero \isactrlbsub \_\isactrlesub ")\ \isanewline
\ \ \ \ \isakeyword{and}\ one\_str\ ("\isasymone \isactrlbsub \_\isactrlesub ")\ \isakeyword{and}\ \isasymFF '\ \isakeyword{and}\ \isasymrho '\ \isakeyword{and}\ b'\ \isanewline
\ \ \ \ \isakeyword{and}\ add\_str'\ ("+'{\kern0pt}'\isactrlbsub \_\isactrlesub ")\ \isakeyword{and}\ mult\_str'\ ("\isasymcdot '{\kern0pt}'\isactrlbsub \_\isactrlesub ")\isanewline
\ \ \ \ \isakeyword{and}\ zero\_str'\ ("\isasymzero '{\kern0pt}'\isactrlbsub \_\isactrlesub ")\ \isakeyword{and}\ one\_str'\ ("\isasymone '{\kern0pt}'\isactrlbsub \_\isactrlesub ")\ +\isanewline
\ \ \isakeyword{fixes}\ fam\_morphisms::\ "'a\ set\ \isasymRightarrow \ ('b\ \isasymRightarrow \ 'c)"\isanewline
\ \ \isakeyword{assumes}\ is\_ring\_morphism:\ \isanewline
\ \ \ \ "\isasymAnd U.\ is\_open\ U\ \isasymLongrightarrow \ ring\_homomorphism\ (fam\_morphisms\ U)\isanewline
\ \ \ \ \ \ \ \ \ \ \ \ \ \ \ \ \ \ \ \ \ \ \ \ \ \ \ \ \ \ (\isasymFF \ U)\ (+\isactrlbsub U\isactrlesub )\ (\isasymcdot \isactrlbsub U\isactrlesub )\ \isasymzero \isactrlbsub U\isactrlesub \ \isasymone \isactrlbsub U\isactrlesub \isanewline
\ \ \ \ \ \ \ \ \ \ \ \ \ \ \ \ \ \ \ \ \ \ \ \ \ (\isasymFF '\ U)\ (+'\isactrlbsub U\isactrlesub )\ (\isasymcdot '\isactrlbsub U\isactrlesub )\ \isasymzero '\isactrlbsub U\isactrlesub \ \isasymone '\isactrlbsub U\isactrlesub "\isanewline
\ \ \ \ \isakeyword{and}\ comm\_diagrams:\ \isanewline
\ \ \ \ "\isasymAnd U\ V\ x.\ \isasymlbrakk is\_open\ U;\ is\_open\ V;\ V\ \isasymsubseteq \ U;x\ \isasymin \ \isasymFF \ U\ \isasymrbrakk \isanewline
\ \ \ \ \ \ \ \ \ \ \ \ \ \ \isasymLongrightarrow \ (\isasymrho '\ U\ V\ \isasymcirc \ fam\_morphisms\ U)\ x\ \isanewline
\ \ \ \ \ \ \ \ \ \ \ \ \ \ \ \ \ \ \ \ \ \ \ \ \ \ \ \ \ \ =\ (fam\_morphisms\ V\ \isasymcirc \ \isasymrho \ U\ V)\ x"
\end{isabelle}

The notion of composition between presheaves is straightforward and we check in Isabelle without any difficulty that the composition is again a presheaf of rings. \\
Indeed,	let $\phi: \mathscr{F} \rightarrow \mathscr{G}$ (\textit{resp.} $\psi: \mathscr{G} \rightarrow \mathscr{H}$) be a morphism of presheaves of rings on a topological space $X$. One defines the composition $\psi \circ \phi$ from $\mathscr{F}$ to $\mathscr{H}$ as 
	\[
	(\psi \circ \phi)_U \coloneqq \psi_U \circ \phi_U
	\]
	for every open subset $U$ of $X$.

% snippet of code to be inserted here for the lemma that states the composition is again a presheaf, namely the statement of `lemma comp_of_presheaves` of Comm_Ring_Theory.thy
\begin{isabelle}
\isacommand{lemma}\ comp\_of\_presheaves:\isanewline
\ \ \isakeyword{assumes}\ "morphism\_presheaves\_of\_rings\ X\ is\_open\ \isasymFF \ \isasymrho \ b\ \isanewline
\ \ \ \ \ \ \ \ \ \ \ \ add\_str\ mult\_str\ zero\_str\ one\_str\ \isasymFF '\ \isasymrho '\ b'\ \isanewline
\ \ \ \ \ \ \ \ \ \ \ \ \ \ \ \ add\_str'\ mult\_str'\ zero\_str'\ one\_str'\ \isasymphi "\isanewline
\ \ \ \ \isakeyword{and}\ "morphism\_presheaves\_of\_rings\ X\ is\_open\ \isasymFF '\ \isasymrho '\ b'\isanewline
\ \ \ \ \ \ \ \ \ \ add\_str'\ mult\_str'\ zero\_str'\ one\_str'\ \isasymFF '{\kern0pt}'\ \isasymrho '{\kern0pt}'\ \isanewline
\ \ \ \ \ \ \ \ \ \ b'{\kern0pt}'\ add\_str'{\kern0pt}'\ mult\_str'{\kern0pt}'\ zero\_str'{\kern0pt}'\ one\_str'{\kern0pt}'\ \isasymphi '"\isanewline
\ \ \isakeyword{shows}\ "morphism\_presheaves\_of\_rings\ X\ is\_open\ \isasymFF \ \isasymrho \ b\ add\_str\isanewline
\ \ \ \ \ \ mult\_str\ zero\_str\ one\_str\ \isasymFF '{\kern0pt}'\ \isasymrho '{\kern0pt}'\ b'{\kern0pt}'\ add\_str'{\kern0pt}'\ \isanewline
\ \ \ \ \ \ mult\_str'{\kern0pt}'\ zero\_str'{\kern0pt}'\ one\_str'{\kern0pt}'\ (\isasymlambda U.\ (\isasymphi '\ U\ \isasymcirc \ \isasymphi \ U\ \isasymdown \ \isasymFF \ U))"
\end{isabelle}

\begin{remark}
	Given a presheaf $\mathscr{F}$ on $X$, let us define $(1_{\mathscr{F}})_U$ as the identity morphism of $\mathscr{F}(U)$ for every open subset $U$ of $X$. $1_{\mathscr{F}}$ is obviously a morphism of presheaves from $\mathscr{F}$ to itself. A morphism $\phi: \mathscr{F} \rightarrow \mathscr{G}$ of presheaves of rings is an isomorphism if and only if there exists a morphism $\psi: \mathscr{G} \rightarrow \mathscr{F}$ such that $\psi \circ \phi = 1_{\mathscr{F}}$ and $\phi \circ \psi = 1_{\mathscr{G}}$.  
\end{remark}

% snippet of code to be inserted `locale iso_presheaves_of_rings` of Comm_Ring_Theory.thy
\begin{isabelle}
\isacommand{locale}\ iso\_presheaves\_of\_rings\ \isanewline
\ \ =\ mor:morphism\_presheaves\_of\_rings\ +\ \isanewline
\ \ \isakeyword{assumes}\ is\_inv:\ "\isasymexists \isasympsi .\ morphism\_presheaves\_of\_rings\ X\ is\_open\ \isanewline
\ \ \ \ \isasymFF '\ \isasymrho '\ b'\ add\_str'\ mult\_str'\ zero\_str'\ one\_str'\ \isasymFF \ \isasymrho \ b\ \isanewline
\ \ \ \ add\_str\ mult\_str\ zero\_str\ one\_str\ \isasympsi \ \isasymand \ (\isasymforall U.\ is\_open\ U\ \isanewline
\ \ \ \ \ \ \ \ \ \ \isasymlongrightarrow \ (\isasymforall x\ \isasymin \ (\isasymFF '\ U).\ (fam\_morphisms\ U\ \isasymcirc \ \isasympsi \ U)\ x\ =\ x)\ \isanewline
\ \ \ \ \ \ \ \ \ \ \ \ \ \ \ \ \isasymand \ (\isasymforall x\ \isasymin \ (\isasymFF \ U).\ (\isasympsi \ U\ \isasymcirc \ fam\_morphisms\ U)\ x\ =\ x))"
\end{isabelle}

Now, we introduce the notion of a \emph{sheaf of rings}. 			

\begin{definition}[sheaf of rings]
	A sheaf of rings on a topological space $X$ is a presheaf on $X$ that satisfies the following additional properties:
	\begin{enumerate}
	%\setcounter{enumi}{3}
		\item[(locality)] Given an open set $U$, $\lbrace V_i \rbrace$ an open covering of $U$ and $s \in \mathscr{F}(U)$ such that $s \restriction V_i = 0$ for all $i$, then one has $s = 0$.
		\item[(glueing)] Given an open set $U$, $\lbrace V_i \rbrace$ an open covering of $U$ and elements $s_i \in \mathscr{F}(V_i)$ satisfying the property $s_i \restriction V_i \cap V_j = s_j \restriction V_i \cap V_j$ for all $i$ and $j$, then there exists an element $s \in \mathscr{F}(U)$ such that $s \restriction V_i = s_i$ for all $i$.  
	\end{enumerate}	
\end{definition}

% snippet of code to be inserted for `locale sheaf_of_rings` of Comm_Ring_Theory.thy
\begin{isabelle}
\isacommand{locale}\ sheaf\_of\_rings\ =\ presheaf\_of\_rings\ +\isanewline
\ \ \isakeyword{assumes}\ locality:\ "\isasymAnd U\ I\ V\ s.\ open\_cover\_of\_open\_subset\ S\ \isanewline
\ \ \ \ \ \ \ \ \ \ is\_open\ U\ I\ V\ \isasymLongrightarrow \ (\isasymAnd i.\ i\isasymin I\ \isasymLongrightarrow \ V\ i\ \isasymsubseteq \ U)\ \isasymLongrightarrow \ s\ \isasymin \ \isasymFF \ U\isanewline
\ \ \ \ \ \ \ \ \ \ \ \isasymLongrightarrow \ (\isasymAnd i.\ i\isasymin I\ \isasymLongrightarrow \ \isasymrho \ U\ (V\ i)\ s\ =\ \isasymzero \isactrlbsub (V\ i)\isactrlesub )\ \isasymLongrightarrow \ s\ =\ \isasymzero \isactrlbsub U\isactrlesub "\isanewline
\ \ \ \ \isakeyword{and}\ glueing:\ \ \isanewline
\ \ \ \ \ \ "\isasymAnd U\ I\ V\ s.\ open\_cover\_of\_open\_subset\ S\ is\_open\ U\ I\ V\ \isanewline
\ \ \ \ \ \ \ \ \isasymLongrightarrow \ (\isasymforall i.\ i\isasymin I\ \isasymlongrightarrow \ V\ i\ \isasymsubseteq \ U\ \isasymand \ s\ i\ \isasymin \ \isasymFF \ (V\ i))\ \isanewline
\ \ \ \ \ \ \ \ \isasymLongrightarrow \ (\isasymAnd i\ j.\ i\isasymin I\ \isasymLongrightarrow \ j\isasymin I\ \isasymLongrightarrow \ \isasymrho \ (V\ i)\ (V\ i\ \isasyminter \ V\ j)\ (s\ i)\isanewline
\ \ \ \ \ \ \ \ \ \ \ \ \ \ \ \ \ \ =\ \isasymrho \ (V\ j)\ (V\ i\ \isasyminter \ V\ j)\ (s\ j))\ \isanewline
\ \ \ \ \ \ \ \ \isasymLongrightarrow \ (\isasymexists t.\ t\ \isasymin \ \isasymFF \ U\ \isasymand \ (\isasymforall i.\ i\isasymin I\ \isasymlongrightarrow \ \isasymrho \ U\ (V\ i)\ t\ =\ s\ i))"
\end{isabelle}
 
At this point, let me quote the mathematician Kevin Buzzard:
\begin{quote}
	I am not sure that it is even possible to write this code in Isabelle/HOL in such a way that it will run in finite time, [\dots], and a sheaf is a dependent type, and your clever HOL workarounds will not let you use typeclasses [\dots]\footnote{\url{https://xenaproject.wordpress.com/2020/02/09/where-is-the-fashionable-mathematics/}. See the section entitled \textit{Isabelle/HOL users}.}
\end{quote}

We do not use old-fashioned type classes, we do use modern locales. We encapsulated within the locale dedicated to a preasheaf of rings $\mathscr{F}$ the ring structures on the underlying sets $\mathscr{F}(U)$'s simply using the functional \textit{add\_str} (\textit{resp.} \textit{mult\_str}, \textit{zero\_str}, \textit{one\_str}) that takes an open subset $U$ and outputs the multiplication (\textit{resp.} the addition, the additive unit, the multiplicative unit) on the set $\mathscr{F}(U)$. The resulting formalization of sheaves of rings in Isabelle is not only possible, but it is concise (6 lines), very neat and tidy, almost transparent with respect to the pen-and-paper definition. \\
It will not be any more difficult to define (pre)sheaves valued in any category of structured sets (groups, modules \dots) or even to define in Isabelle a general notion of $\mathscr{C}$-valued (pre)sheaves by embarking the category $\mathscr{C}$ as part of the data within the locale.  	


A morphism (\textit{resp.} an isomorphism) of sheaves of rings is nothing but a morphism (\textit{resp.} an isomorphism) of presheaves of rings, hence the following locale in Isabelle.

% snippet of code to be inserted `locale morphism_sheaves_of_rings` of Comm_Ring_Theory.thy
\begin{isabelle}
\isacommand{locale}\ morphism\_sheaves\_of\_rings\ =\ morphism\_presheaves\_of\_rings
\end{isabelle}

Let $X$ be a topological space, $\mathscr{F}$ a sheaf of rings on $X$ and $U$ an open subset of $X$,  one proves that $\mathscr{F}|_U$, defined by $\mathscr{F}|_U(V) \coloneqq \mathscr{F}(U \cap V)$, is a sheaf of rings on $U$ for the induced topology. \\
We encapsulate the relevant mathematical context (the ``let be'' part) in a dedicated locale and formalize within this locale the relevant definitions and we eventually prove  that the \emph{induced sheaf $\mathscr{F}|_U$} is indeed a sheaf of rings.

% snippet of code to be inserted here `locale ind_sheaf` and the following interpretation and definitions up to `definition ind_one_str` and then the statement of `lemma ind_sheaf_is_sheaf` 		
\begin{isabelle}
\isacommand{locale}\ ind\_sheaf\ =\ sheaf\_of\_rings\ +\isanewline
\ \ \isakeyword{fixes}\ U::\ "'a\ set"\isanewline
\ \ \isakeyword{assumes}\ is\_open\_subset:\ "is\_open\ U"\isanewline
\isakeyword{begin}\isanewline
\isanewline
\isacommand{interpretation}\ it:\ ind\_topology\ S\ is\_open\ U\isanewline

\isacommand{definition}\ ind\_sheaf::\ "'a\ set\ \isasymRightarrow \ 'b\ set"\isanewline
\ \ \isakeyword{where}\ "ind\_sheaf\ V\ \isasymequiv \ \isasymFF \ (U\ \isasyminter \ V)"\isanewline
\isanewline
\isacommand{definition}\ ind\_ring\_morphisms::\ "'a\ set\ \isasymRightarrow \ 'a\ set\ \isasymRightarrow \ ('b\ \isasymRightarrow \ 'b)"\isanewline
\ \ \isakeyword{where}\ "ind\_ring\_morphisms\ V\ W\ \isasymequiv \ \isasymrho \ (U\ \isasyminter \ V)\ (U\ \isasyminter \ W)"\isanewline
\isanewline
\isacommand{definition}\ ind\_add\_str::\ "'a\ set\ \isasymRightarrow \ ('b\ \isasymRightarrow \ 'b\ \isasymRightarrow \ 'b)"\isanewline
\ \ \isakeyword{where}\ "ind\_add\_str\ V\ \isasymequiv \ \isasymlambda x\ y.\ +\isactrlbsub (U\ \isasyminter \ V)\isactrlesub \ x\ y"\isanewline
\isanewline
\isacommand{definition}\ ind\_mult\_str::\ "'a\ set\ \isasymRightarrow \ ('b\ \isasymRightarrow \ 'b\ \isasymRightarrow \ 'b)"\isanewline
\ \ \isakeyword{where}\ "ind\_mult\_str\ V\ \isasymequiv \ \isasymlambda x\ y.\ \isasymcdot \isactrlbsub (U\ \isasyminter \ V)\isactrlesub \ x\ y"\isanewline
\isanewline
\isacommand{definition}\ ind\_zero\_str::\ "'a\ set\ \isasymRightarrow \ 'b"\isanewline
\ \ \isakeyword{where}\ "ind\_zero\_str\ V\ \isasymequiv \ \isasymzero \isactrlbsub (U\isasyminter V)\isactrlesub "\isanewline
\isanewline
\isacommand{definition}\ ind\_one\_str::\ "'a\ set\ \isasymRightarrow \ 'b"\isanewline
\ \ \isakeyword{where}\ "ind\_one\_str\ V\ \isasymequiv \ \isasymone \isactrlbsub (U\isasyminter V)\isactrlesub "\isanewline
\isanewline
\isacommand{lemma}\ ind\_sheaf\_is\_sheaf:\isanewline
\ \ "sheaf\_of\_rings\ U\ it.ind\_is\_open\ ind\_sheaf\ ind\_ring\_morphisms\ \isanewline
\ \ \ \ \ \ \ \ \ \ \ \ b\ ind\_add\_str\ ind\_mult\_str\ ind\_zero\_str\ ind\_one\_str"\isanewline

\isacommand{end}
\end{isabelle}

We introduce a second construction that takes as input a given sheaf of rings and outputs a new sheaf of rings. Let $f: X \rightarrow Y$ be a continuous map between topological spaces and $\mathscr{F}$ a sheaf of rings on $X$. Given an open subset $V$ of $Y$, define $(f_{*} \mathscr{F})(V)$ to be $\mathscr{F} (f^{-1}(V))$. We then prove $f_{*} \mathscr{F}$, the so-called \emph{direct image of  $\mathscr{F}$}, is a sheaf of rings on the topological space $Y$.  

% snippet of code to be inserted `im_sheaf` of Comm_Ring_Theory.thy plus the following definitions up to `one_im_sheaf` and then the `lemma im_sheaf_is_sheaf` of Comm_Ring_Theory.thy
\begin{isabelle}
\isacommand{locale}\ im\_sheaf\ =\ sheaf\_of\_rings\ +\ continuous\_map\isanewline
\isakeyword{begin}\isanewline
\isanewline
\isacommand{definition}\ im\_sheaf::\ "'c\ set\ =>\ 'b\ set"\isanewline
\ \ \isakeyword{where}\ "im\_sheaf\ V\ \isasymequiv \ \isasymFF \ (f\isactrlsup \isasyminverse \ S\ V)"\isanewline
\isanewline
\isacommand{definition}\ im\_sheaf\_morphisms::\ "'c\ set\ \isasymRightarrow \ 'c\ set\ \isasymRightarrow \ ('b\ \isasymRightarrow \ 'b)"\isanewline
\ \ \isakeyword{where}\ "im\_sheaf\_morphisms\ U\ V\ \isasymequiv \ \isasymrho \ (f\isactrlsup \isasyminverse \ S\ U)\ (f\isactrlsup \isasyminverse \ S\ V)"\isanewline
\isanewline
\isacommand{definition}\ add\_im\_sheaf::\ "'c\ set\ \isasymRightarrow \ 'b\ \isasymRightarrow \ 'b\ \isasymRightarrow \ 'b"\isanewline
\ \ \isakeyword{where}\ "add\_im\_sheaf\ \isasymequiv \ \isasymlambda V\ x\ y.\ +\isactrlbsub (f\isactrlsup \isasyminverse \ S\ V)\isactrlesub \ x\ y"\isanewline
\isanewline
\isacommand{definition}\ mult\_im\_sheaf::\ "'c\ set\ \isasymRightarrow \ 'b\ \isasymRightarrow \ 'b\ \isasymRightarrow \ 'b"\isanewline
\ \ \isakeyword{where}\ "mult\_im\_sheaf\ \isasymequiv \ \isasymlambda V\ x\ y.\ \isasymcdot \isactrlbsub (f\isactrlsup \isasyminverse \ S\ V)\isactrlesub \ x\ y"\isanewline
\isanewline
\isacommand{definition}\ zero\_im\_sheaf::\ "'c\ set\ \isasymRightarrow \ 'b"\isanewline
\ \ \isakeyword{where}\ "zero\_im\_sheaf\ \isasymequiv \ \isasymlambda V.\ \isasymzero \isactrlbsub (f\isactrlsup \isasyminverse \ S\ V)\isactrlesub "\isanewline
\isanewline
\isacommand{definition}\ one\_im\_sheaf::\ "'c\ set\ \isasymRightarrow \ 'b"\isanewline
\ \ \isakeyword{where}\ "one\_im\_sheaf\ \isasymequiv \ \isasymlambda V.\ \isasymone \isactrlbsub (f\isactrlsup \isasyminverse \ S\ V)\isactrlesub "\isanewline
\isanewline
\isacommand{lemma}\ im\_sheaf\_is\_sheaf:\isanewline
\ \ "sheaf\_of\_rings\ S'\ is\_open'\ im\_sheaf\ im\_sheaf\_morphisms\ b\isanewline
\ \ \ \ \ \ \ \ \ \ add\_im\_sheaf\ mult\_im\_sheaf\ zero\_im\_sheaf\ one\_im\_sheaf"\isanewline
\isacommand{end}
\end{isabelle}

As can been seen in the snippet of code above, we encapsulate again the context of the construction in a dedicated locale and this lets us formalize the relevant definitions in this context and we eventually prove the direct image of a sheaf is again a sheaf. 

Next, we take a detour into the localization of commutative rings. This construction was already formalized by the first author in \cite{Localization_Ring-AFP}, but we give a more concise formalization here since the first author gained a better knowledge of the locale mechanism in the meantime. \\
Let $S$ be a multiplicative submonoid of $R$. We consider pairs $(r, s)$ with $r \in R$ and $s \in S$ and we define the following relation on them
	\[
	(r, s) \sim (r', s')
	\]
if and only if there exists $s_1 \in S$ such that $s_1(s' r - s r') = 0$. One checks that the relation $\sim$ is an equivalence relation. The equivalence class of a pair $(r, s)$ is denoted by $r/s$ and the set of equivalence classes is denoted by $S^{-1} R$. \\
We define the following addition on $S^{-1} R$.
	\[
	\frac{r}{s} + \frac{r'}{s'} = \frac{r s' + r' s}{s s'}
	\]
We also define the following multiplication on $S^{-1} R$. 
	\[
	\frac{r}{s} \times \frac{r'}{s'} = \frac{r r'}{s s'}
	\]
One checks $(S^{-1} R, 0/1, +, 1/1, \times)$ is a ring. The new ring $S^{-1} R$ is called the \emph{quotient ring of $R$ by $S$}. \\
This construction translates into the following very legible locale ending with a proof that the quotient ring is indeed a ring.

% snippet of code to be inserted for `locale quotient_ring`	of Comm_Ring_Theory.thy including the definitions `rel`, `frac`, the lemma `add_rel_frac`, `mult_rel_frac`, the definitions `zero_rel`, `one_rel`
% and the sublocale that proves the quotient ring is a commutative ring.

\begin{isabelle}
\isacommand{locale}\ quotient\_ring\ =\ comm:comm\_ring\ R\ "(+)"\ "(\isasymcdot )"\ "\isasymzero "\ "\isasymone "\ \isanewline
\ \ \ \ \ \ \ \ \ \ +\ submonoid\ S\ R\ "(\isasymcdot )"\ "\isasymone "\isanewline
\ \ \ \ \ \ \isakeyword{for}\ S\ \isakeyword{and}\ R\ \isakeyword{and}\ addition\ (\isakeyword{infixl}\ "+"\ 65)\ \isanewline
\ \ \ \ \ \ \ \ \isakeyword{and}\ multiplication\ (\isakeyword{infixl}\ "\isasymcdot "\ 70)\ \isakeyword{and}\ zero\ ("\isasymzero ")\ \isakeyword{and}\ unit\ ("\isasymone ")\isanewline
\isakeyword{begin}\isanewline
\isanewline
\isacommand{definition}\ rel::\ "('a\ \isasymtimes \ 'a)\ \isasymRightarrow \ ('a\ \isasymtimes \ 'a)\ \isasymRightarrow \ bool"\ (\isakeyword{infix}\ "\isasymsim "\ 80)\isanewline
\ \ \isakeyword{where}\ "x\ \isasymsim \ y\ \isasymequiv \ \isasymexists s1.\ s1\ \isasymin \ S\ \isasymand \ s1\ \isasymcdot \ (snd\ y\ \isasymcdot \ fst\ x\ -\ snd\ x\ \isasymcdot \ fst\ y)\ =\ \isasymzero "\isanewline
\isanewline
\isacommand{interpretation}\ rel:\ equivalence\ "R\ \isasymtimes \ S"\ "\{(x,y)\ \isasymin \ (R\isasymtimes S)\isasymtimes (R\isasymtimes S).\ x\ \isasymsim \ y\}"\isanewline
\isanewline
\isacommand{definition}\ frac::\ "'a\ \isasymRightarrow \ 'a\ \isasymRightarrow \ ('a\ \isasymtimes \ 'a)\ set"\ (\isakeyword{infixl}\ "'/"\ 75)\isanewline
\ \ \isakeyword{where}\ "r\ /\ s\ \isasymequiv \ rel.Class\ (r,\ s)"\isanewline
\isanewline
\isacommand{lemma}\ add\_rel\_frac:\isanewline
\ \ \isakeyword{assumes}\ "(r,s)\ \isasymin \ R\ \isasymtimes \ S"\ \isakeyword{and}\ "(r',s')\isasymin \ R\ \isasymtimes \ S"\isanewline
\ \ \isakeyword{shows}\ "add\_rel\ (r/s)\ (r'/s')\ =\ (r\isasymcdot s'\ +\ r'\isasymcdot s)\ /\ (s\isasymcdot s')"
\isanewline
\isanewline
\isacommand{lemma}\ mult\_rel\_frac:\isanewline
\ \ \isakeyword{assumes}\ "(r,s)\ \isasymin \ R\ \isasymtimes \ S"\ \isakeyword{and}\ "(r',s')\isasymin \ R\ \isasymtimes \ S"\isanewline
\ \ \isakeyword{shows}\ "mult\_rel\ (r/s)\ (r'/s')\ =\ (r\isasymcdot \ r')\ /\ (s\isasymcdot s')"
\isanewline
\isanewline
\isacommand{definition}\ zero\_rel::"('a\ \isasymtimes \ 'a)\ set"\ \isakeyword{where}\isanewline
\ \ "zero\_rel\ =\ frac\ \isasymzero \ \isasymone "\isanewline
\isanewline
\isacommand{definition}\ one\_rel::"('a\ \isasymtimes \ 'a)\ set"\ \isakeyword{where}\isanewline
\ \ "one\_rel\ =\ frac\ \isasymone \ \isasymone "\isanewline
\isanewline
\isacommand{sublocale}\ comm\_ring\ carrier\_quotient\_ring\ add\_rel\ mult\_rel\ zero\_rel\ one\_rel\isanewline
\isanewline
\isacommand{end}
\end{isabelle}

Remember that the complement of a prime ideal $\mathfrak{p}$ in $R$ is a multiplicative submonoid, hence we can apply the previous construction with $S \coloneqq R - \mathfrak{p}$. The resulting ring  $S^{-1} R$ is called the \emph{local ring of $R$ at $\mathfrak{p}$}. \\

% snippet of code to be inserted for the subsection <Local Rings at Prime Ideals> up to the definition `one_local_ring_at`
\begin{isabelle}
\isacommand{context}\ pr\_ideal\isanewline
\isakeyword{begin}\isanewline
\isanewline
\isacommand{lemma}\ submonoid\_pr\_ideal:\isanewline
\ \ \isakeyword{shows}\ "submonoid\ (R\ \isasymsetminus \ I)\ R\ (\isasymcdot )\ \isasymone "\isanewline
\isacommand{proof}\isanewline
\ \ \isacommand{show}\ "a\ \isasymcdot \ b\ \isasymin \ R\isasymsetminus I"\ \isakeyword{if}\ "a\ \isasymin \ R\isasymsetminus I"\ "b\ \isasymin \ R\isasymsetminus I"\ \isakeyword{for}\ a\ b\isanewline
\ \ \ \ \isacommand{using}\ that\ \isacommand{by}\ (metis\ Diff\_iff\ absorbent\ comm.multiplicative.composition\_closed)\isanewline
\ \ \isacommand{show}\ "\isasymone \ \isasymin \ R\isasymsetminus I"\isanewline
\ \ \ \ \isacommand{using}\ ideal.ideal(2)\ ideal\_axioms\ pr\_ideal.carrier\_neq\ pr\_ideal\_axioms\ \isanewline
\ \ \ \ \isacommand{by}\ fastforce\isanewline
\isacommand{qed}\ auto%
\isanewline
\isanewline
\isacommand{interpretation}\ local:quotient\_ring\ "(R\ \isasymsetminus \ I)"\ R\ "(+)"\ "(\isasymcdot )"\ \isasymzero \ \isasymone \isanewline
\ \ \isacommand{apply}\ intro\_locales\isanewline
\ \ \isacommand{by}\ (meson\ submonoid\_def\ submonoid\_pr\_ideal)

\isanewline
\isacommand{definition}\ carrier\_local\_ring\_at::\ "('a\ \isasymtimes \ 'a)\ set\ set"\isanewline
\ \ \isakeyword{where}\ "carrier\_local\_ring\_at\ \isasymequiv \ (R\ \isasymsetminus \ I)\isactrlsup \isasyminverse \ R\isactrlbsub (+)\ (\isasymcdot )\ \isasymzero \isactrlesub "\isanewline
\isanewline
\isacommand{definition}\ add\_local\_ring\_at::\ "('a\ \isasymtimes \ 'a)\ set\ \isasymRightarrow \ ('a\ \isasymtimes \ 'a)\ set\ \isasymRightarrow \ ('a\ \isasymtimes \ 'a)\ set"\isanewline
\ \ \isakeyword{where}\ "add\_local\_ring\_at\ \isasymequiv \ local.add\_rel\ "\isanewline
\isanewline
\isacommand{definition}\ mult\_local\_ring\_at::\ "('a\ \isasymtimes \ 'a)\ set\ \isasymRightarrow \ ('a\ \isasymtimes \ 'a)\ set\ \isasymRightarrow \ ('a\ \isasymtimes \ 'a)\ set"\isanewline
\ \ \isakeyword{where}\ "mult\_local\_ring\_at\ \isasymequiv \ local.mult\_rel\ "\isanewline
\isanewline
\isacommand{definition}\ uminus\_local\_ring\_at::\ "('a\ \isasymtimes \ 'a)\ set\ \isasymRightarrow \ ('a\ \isasymtimes \ 'a)\ set"\isanewline
\ \ \isakeyword{where}\ "uminus\_local\_ring\_at\ \isasymequiv \ local.uminus\_rel\ "\isanewline
\isanewline
\isacommand{definition}\ zero\_local\_ring\_at::\ "('a\ \isasymtimes \ 'a)\ set"\isanewline
\ \ \isakeyword{where}\ "zero\_local\_ring\_at\ \isasymequiv \ local.zero\_rel"\isanewline
\isanewline
\isacommand{definition}\ one\_local\_ring\_at::\ "('a\ \isasymtimes \ 'a)\ set"\isanewline
\ \ \isakeyword{where}\ "one\_local\_ring\_at\ \isasymequiv \ local.one\_rel"\isanewline
\isanewline
\isacommand{end}
\end{isabelle}

This allows us to introduce the next construction.	\\
Let $U$ be an open subset of $\text{Spec}\,R$. We define $\mathscr{O}_{\spec} (U)$ to be the set of all functions 
	\[
	s: U \rightarrow \coprod \limits_{\mathfrak{p} \in U} R_{\mathfrak{p}}
	\]
such that $s(\mathfrak{p}) \in R_{\mathfrak{p}}$ for every $\mathfrak{p} \in U$ and such that for every $\mathfrak{p} \in U$, there exist a neighborhood $V$ of $\mathfrak{p}$, contained in $U$, and elements $r, f \in R$, such that for each $\mathfrak{q} \in V$, $f \notin \mathfrak{q}$ and $s(\mathfrak{q}) = r/f$. \\
We endow $\mathscr{O}_{\spec} (U)$ with the sum and product of such functions and we take for the additive unit (\textit{resp.} multiplicative unit) the function that maps each $\mathfrak{p}$ on the additive unit (\textit{resp.} multiplicative unit) of $R_{\mathfrak{p}}$. \\
Last, if $V \subseteq U$, then the morphism of rings $\mathscr{O}_{\spec} (U) \rightarrow \mathscr{O}_{\spec} (V)$ maps each $s \in \mathscr{O}_{\spec} (U)$ on its restriction to $V$. \\
One checks $\mathscr{O}_{\spec} $ is a sheaf of rings on $\text{Spec}\,R$ and the pair $(\text{Spec}\,R, \mathscr{O}_{\spec} )$ is called the \emph{spectrum of the ring $R$}. \\
This last construction translates smoothly in Isabelle. 

% snippet of code here for the defintions `is_locally_frac`, `is_regular`, `sheaf_spec`, `add_sheaf_spec`, `mult_sheaf_spec`, `zero_sheaf_spec`, `one_sheaf_spec`, `sheaf_spec_morphisms` and finally the statement of `lemma sheaf_spec_is_sheaf`.

\begin{isabelle}
\isacommand{definition}\ (\isakeyword{in}\ comm\_ring)\ \isanewline
\ \ \ \ is\_locally\_frac::\ "('a\ set\ \isasymRightarrow \ ('a\ \isasymtimes \ 'a)\ set)\ \isasymRightarrow \ 'a\ set\ set\ \isasymRightarrow \ bool"\isanewline
\ \ \ \ \isakeyword{where}\ "is\_locally\_frac\ s\ V\ \isasymequiv \ (\isasymexists r\ f.\ r\ \isasymin \ R\ \isasymand \ f\ \isasymin \ R\ \isasymand \ (\isasymforall \isasymqq \ \isasymin \ V.\ \isanewline
\ \ \ \ \ \ \ \ \ \ \ \ f\ \isasymnotin \ \isasymqq \ \isasymand \ s\ \isasymqq \ =\ quotient\_ring.frac\ (R\ \isasymsetminus \ \isasymqq )\ R\ (+)\ (\isasymcdot )\ \isasymzero \ r\ f))"\isanewline
\isanewline
\isacommand{definition}\ (\isakeyword{in}\ comm\_ring)\ is\_regular\isanewline
\ \ \ \ \ \ \ \ \ \ \ \ \ \ ::\ "('a\ set\ \isasymRightarrow \ ('a\ \isasymtimes \ 'a)\ set)\ \isasymRightarrow \ 'a\ set\ set\ \isasymRightarrow \ bool"\isanewline
\ \ \isakeyword{where}\ "is\_regular\ s\ U\ \isasymequiv \ \isasymforall \isasympp .\ \isasympp \ \isasymin \ U\ \isasymlongrightarrow \ \isanewline
\ \ (\isasymexists V.\ is\_zariski\_open\ V\ \isasymand \ V\ \isasymsubseteq \ U\ \isasymand \ \isasympp \ \isasymin \ V\ \isasymand \ (is\_locally\_frac\ s\ V))"\isanewline
\isanewline
\isacommand{definition}\ (\isakeyword{in}\ comm\_ring)\ sheaf\_spec\isanewline
\ \ \ \ ::\ "'a\ set\ set\ \isasymRightarrow \ ('a\ set\ \isasymRightarrow \ ('a\ \isasymtimes \ 'a)\ set)\ set"\ ("\isasymO \ \_"\ [90]\ 90)\isanewline
\ \ \isakeyword{where}\ "\isasymO \ U\ \isasymequiv \ \{s\isasymin (\isasymPi \isactrlsub E\ \isasympp \isasymin U.\ (R\isactrlbsub \isasympp \ (+)\ (\isasymcdot )\ \isasymzero \isactrlesub )).\ is\_regular\ s\ U\}"\isanewline
\isanewline
\isacommand{definition}\ (\isakeyword{in}\ comm\_ring)\ add\_sheaf\_spec\isanewline
\ \ \ \ ::\ "('a\ set)\ set\ \isasymRightarrow \ ('a\ set\ \isasymRightarrow \ ('a\ \isasymtimes \ 'a)\ set)\ \isanewline
\ \ \ \ \ \ \ \ \ \ \isasymRightarrow \ ('a\ set\ \isasymRightarrow \ ('a\ \isasymtimes \ 'a)\ set)\ \isasymRightarrow \ ('a\ set\ \isasymRightarrow \ ('a\ \isasymtimes \ 'a)\ set)"\isanewline
\ \ \ \ \isakeyword{where}\ "add\_sheaf\_spec\ U\ s\ s'\ \isasymequiv \ \isanewline
\ \ \ \ \ \ \ \ \ \ \isasymlambda \isasympp \isasymin U.\ quotient\_ring.add\_rel\ (R\ \isasymsetminus \ \isasympp )\ R\ (+)\ (\isasymcdot )\ \isasymzero \ (s\ \isasympp )\ (s'\ \isasympp )"\isanewline
\isanewline
\isacommand{definition}\ (\isakeyword{in}\ comm\_ring)\ mult\_sheaf\_spec\isanewline
\ \ \ \ ::\ "('a\ set)\ set\ \isasymRightarrow \ ('a\ set\ \isasymRightarrow \ ('a\ \isasymtimes \ 'a)\ set)\ \isanewline
\ \ \ \ \ \ \ \ \ \ \ \ \isasymRightarrow \ ('a\ set\ \isasymRightarrow \ ('a\ \isasymtimes \ 'a)\ set)\ \isasymRightarrow \ ('a\ set\ \isasymRightarrow \ ('a\ \isasymtimes \ 'a)\ set)"\isanewline
\ \ \ \ \isakeyword{where}\ "mult\_sheaf\_spec\ U\ s\ s'\ \isasymequiv \ \isanewline
\ \ \ \ \ \ \ \ \ \isasymlambda \isasympp \isasymin U.\ quotient\_ring.mult\_rel\ (R\ \isasymsetminus \ \isasympp )\ R\ (+)\ (\isasymcdot )\ \isasymzero \ (s\ \isasympp )\ (s'\ \isasympp )"\isanewline
\isanewline
\isacommand{definition}\ (\isakeyword{in}\ comm\_ring)\ zero\_sheaf\_spec\isanewline
\ \ \ \ ::\ "'a\ set\ set\ \isasymRightarrow \ ('a\ set\ \isasymRightarrow \ ('a\ \isasymtimes \ 'a)\ set)"\isanewline
\ \ \ \ \isakeyword{where}\ "zero\_sheaf\_spec\ U\ \isasymequiv \ \isanewline
\ \ \ \ \ \ \ \ \ \ \ \ \ \ \ \ \ \ \ \isasymlambda \isasympp \isasymin U.\ quotient\_ring.zero\_rel\ (R\ \isasymsetminus \ \isasympp )\ R\ (+)\ (\isasymcdot )\ \isasymzero \ \isasymone "\isanewline
\isanewline
\isacommand{definition}\ (\isakeyword{in}\ comm\_ring)\ one\_sheaf\_spec\isanewline
\ \ \ \ ::\ "'a\ set\ set\ \isasymRightarrow \ ('a\ set\ \isasymRightarrow \ ('a\ \isasymtimes \ 'a)\ set)"\isanewline
\ \ \ \ \isakeyword{where}\ "one\_sheaf\_spec\ U\ \isasymequiv \ \isanewline
\ \ \ \ \ \ \ \ \ \ \ \ \ \ \ \ \ \ \ \ \isasymlambda \isasympp \isasymin U.\ quotient\_ring.one\_rel\ (R\ \isasymsetminus \ \isasympp )\ R\ (+)\ (\isasymcdot )\ \isasymzero \ \isasymone "\isanewline
\isanewline
\isacommand{definition}\ (\isakeyword{in}\ comm\_ring)\ sheaf\_spec\_morphisms\isanewline
\ \ \ \ \ \ \ \ ::\ "'a\ set\ set\ \isasymRightarrow \ 'a\ set\ set\ \isasymRightarrow \ (('a\ set\ \isasymRightarrow \ ('a\ \isasymtimes \ 'a)\ set)\ \isanewline
\ \ \ \ \ \ \ \ \ \ \ \ \ \ \ \ \ \ \ \ \ \ \ \ \ \ \ \ \ \ \ \ \ \ \ \ \ \ \ \ \isasymRightarrow \ ('a\ set\ \isasymRightarrow \ ('a\ \isasymtimes \ 'a)\ set))"\isanewline
\ \ \ \ \ \ \ \ \isakeyword{where}\ "sheaf\_spec\_morphisms\ U\ V\ \isasymequiv \ \isasymlambda s\isasymin (\isasymO \ U).\ restrict\ s\ V"\isanewline
\isanewline
\isacommand{lemma}\ (\isakeyword{in}\ comm\_ring)\ sheaf\_spec\_is\_sheaf:\isanewline
\ \ \isakeyword{shows}\ "sheaf\_of\_rings\ Spec\ is\_zariski\_open\ sheaf\_spec\ \isanewline
\ \ \ \ \ \ \ \ \ \ \ \ sheaf\_spec\_morphisms\ \isasymO b\ add\_sheaf\_spec\ mult\_sheaf\_spec\ \isanewline
\ \ \ \ \ \ \ \ \ \ \ \ \ \ \ \ \ \ \ \ \ \ \ \ \ \ \ \ \ \ \ \ \ \ \ \ \ \ \ \ zero\_sheaf\_spec\ one\_sheaf\_spec"
\end{isabelle}

Next, we introduce ringed spaces and their morphisms.

\begin{definition}[ringed space]
	A ringed space is a pair $(X, \mathscr{O}_X)$, where $X$ is a topological space and $\mathscr{O}_X$ is a sheaf of rings on $X$.
\end{definition}

% snippet of code `locale ringed_space` to be inserted here.

\begin{isabelle}
\isacommand{locale}\ ringed\_space\ =\ sheaf\_of\_rings
\end{isabelle}

\begin{definition}[morphism of ringed spaces]
	A morphism of ringed spaces from $(X, \mathscr{O}_X)$ to $(Y, \mathscr{O}_Y)$ is a pair $(f, \phi_f)$ consisting of a continuous map $f: X \rightarrow Y$ between topological spaces and a morphism $\phi_f: \mathscr{O}_Y \rightarrow f_{*} \mathscr{O}_X$ of sheaves of rings.   
\end{definition}

% snippet of code `locale morphism_ringed-spaces` of Comm_Ring_Theory.thy to be inserted here
\begin{isabelle}
\isacommand{locale}\ morphism\_ringed\_spaces\ =\isanewline
\ \ \ \ \ \ im\_sheaf\ X\ is\_open\isactrlsub X\ \isasymO \isactrlsub X\ \isasymrho \isactrlsub X\ b\ add\_str\isactrlsub X\ mult\_str\isactrlsub X\ zero\_str\isactrlsub X\ one\_str\isactrlsub X\ \isanewline
\ \ \ \ \ \ \ \ \ \ \ \ \ \ \ \ \ \ \ \ \ \ \ \ \ \ \ \ \ \ \ \ \ \ \ \ \ \ \ \ \ \ \ \ \ \ \ \ \ \ \ \ \ \ \ \ Y\ is\_open\isactrlsub Y\ f\ \isanewline
\ \ \ \ \ \ +\ codom:\ ringed\_space\ Y\ is\_open\isactrlsub Y\ \isasymO \isactrlsub Y\ \isasymrho \isactrlsub Y\ d\ add\_str\isactrlsub Y\ mult\_str\isactrlsub Y\ \isanewline
\ \ \ \ \ \ \ \ \ \ \ \ \ \ \ \ \ \ \ \ \ \ \ \ \ \ \ \ \ \ \ \ \ \ \ \ \ \ \ \ \ \ \ \ \ \ \ \ \ \ \ \ zero\_str\isactrlsub Y\ one\_str\isactrlsub Y\isanewline
\ \ \ \ \ \ \isakeyword{for}\ X\ \isakeyword{and}\ is\_open\isactrlsub X\ \isakeyword{and}\ \isasymO \isactrlsub X\ \isakeyword{and}\ \isasymrho \isactrlsub X\ \isakeyword{and}\ b\ \isakeyword{and}\ add\_str\isactrlsub X\ \isakeyword{and}\ mult\_str\isactrlsub X\isanewline
\ \ \ \ \ \ \ \ \ \ \isakeyword{and}\ zero\_str\isactrlsub X\ \isakeyword{and}\ one\_str\isactrlsub X\ \isakeyword{and}\ Y\ \isakeyword{and}\ is\_open\isactrlsub Y\ \isakeyword{and}\ \isasymO \isactrlsub Y\ \isakeyword{and}\ \isasymrho \isactrlsub Y\ \isanewline
\ \ \ \ \ \ \ \ \ \ \isakeyword{and}\ d\ \isakeyword{and}\ add\_str\isactrlsub Y\ \isakeyword{and}\ mult\_str\isactrlsub Y\ \isakeyword{and}\ zero\_str\isactrlsub Y\ \isakeyword{and}\ one\_str\isactrlsub Y\isanewline
\ \ \ \ \ \ \ \ \ \ \isakeyword{and}\ f\ \isanewline
\ \ \ \ \ \ +\ \isakeyword{fixes}\ \isasymphi \isactrlsub f::\ "'c\ set\ \isasymRightarrow \ ('d\ \isasymRightarrow \ 'b)"\isanewline
\ \ \ \ \isakeyword{assumes}\ is\_morphism\_of\_sheaves:\ "morphism\_sheaves\_of\_rings\ Y\ \isanewline
\ \ \ \ \ \ \ \ \ \ \ \ \ \ \ \ is\_open\isactrlsub Y\ \isasymO \isactrlsub Y\ \isasymrho \isactrlsub Y\ d\ add\_str\isactrlsub Y\ mult\_str\isactrlsub Y\ zero\_str\isactrlsub Y\ one\_str\isactrlsub Y\isanewline
\ \ \ \ \ \ \ \ \ \ \ \ \ \ \ \ im\_sheaf\ im\_sheaf\_morphisms\ b\ add\_im\_sheaf\ mult\_im\_sheaf\ \isanewline
\ \ \ \ \ \ \ \ \ \ \ \ \ \ \ \ zero\_im\_sheaf\ one\_im\_sheaf\ \isasymphi \isactrlsub f"
\end{isabelle}

A classic and ubiquitous notion in mathematics is the \emph{direct limit} of a family of structured sets, here a family of rings. \\
Let $X$ be a topological space and $\mathscr{F}$ a presheaf of rings on $X$. The set $I$ will denote a subset of the set of all open subsets of $X$ such that for every $U$ and $V$ in $I$ there exists $W$ in $I$ with $W \subseteq U \cap V$. \\
Given $U, V \in I$, $s \in \mathscr{F}(U)$ and $t \in \mathscr{F}(V)$,  we say that 
$s \sim t$ if and only if there exists $W \in I$ such that $W \subseteq U \cap V$ and $s \restriction W = t \restriction W$. One checks that $\sim$ is an equivalence relation. \\
Now, we consider the quotient of the disjoint union of the $\mathscr{F}(U)$'s.
	\[
	\varinjlim_I \mathscr{F}(U) \coloneqq \coprod_{U \in I} \mathscr{F}(U) \bigg/ \sim 
	\]
Last, we define the following binary operations on $\displaystyle \varinjlim_{I} \mathscr{F}(U)$:
	\begin{align*}
	[(U, s)] + [(V, t)] & = [(W, s \restriction W + t \restriction W)] \\
	[(U, s)] \times [(V, t)] & = [(W, s \restriction W \times t \restriction W)]
	\end{align*}
for some $W \in I$ such that $W \subseteq U \cap V$ and where the symbol $[\_]$ denotes the equivalence class of an element. \\
Assuming $I \neq \emptyset$ and $V \in I$, one can prove that  
	\[
	(\varinjlim_I \mathscr{F}(U), [(V, 0_V)], +, [(V, 1_V)], \times)
	\]
is a ring. 

\begin{definition}[direct limit]
	Let $X$ be a topological space, $\mathscr{F}$ a presheaf of rings on $X$ and $I$ a set of open subsets of $X$. The direct limit of $\mathscr{F}$ over $I$ is the ring $\displaystyle \varinjlim_I \mathscr{F}(U)$, denoted simply $\varinjlim \mathscr{F}(U)$ if $I$ is clear from the context.
\end{definition}

The definition above which could naively appear as a dependent type gives raise to a smooth translation in Isabelle using the locale mechanism.

% snippet of code to be inserted here for `locale direct_lim` and the definitions `rel`, `add_rel`, `mult_rel` and the statement of `lemma exercise_0_35`

\begin{isabelle}
\isacommand{locale}\ direct\_lim\ =\ sheaf\_of\_rings\ +\isanewline
\ \ \isakeyword{fixes}\ I::\ "'a\ set\ set"\isanewline
\ \ \isakeyword{assumes}\ subset\_of\_opens:\ "\isasymAnd U.\ U\ \isasymin \ I\ \isasymLongrightarrow \ is\_open\ U"\isanewline
\ \ \ \ \isakeyword{and}\ has\_lower\_bound:\ "\isasymAnd U\ V.\ \isasymlbrakk \ U\isasymin I;\ V\isasymin I\ \isasymrbrakk \ \isasymLongrightarrow \ \isasymexists W\isasymin I.\ W\ \isasymsubseteq \ U\ \isasyminter \ V"\isanewline
\isakeyword{begin}\isanewline
\isanewline
\isacommand{definition}\ rel::\ "('a\ set\ \isasymtimes \ 'b)\ \isasymRightarrow \ ('a\ set\ \isasymtimes \ 'b)\ \isasymRightarrow \ bool"\ (\isakeyword{infix}\ "\isasymsim "\ 80)\isanewline
\ \ \isakeyword{where}\ "x\ \isasymsim \ y\ \isasymequiv \ (fst\ x\ \isasymin \ I\ \isasymand \ fst\ y\ \isasymin \ I)\ \isasymand \ (snd\ x\ \isasymin \ \isasymFF \ (fst\ x)\ \isanewline
\ \ \ \ \ \ \ \ \ \ \ \ \ \ \ \ \ \isasymand \ snd\ y\ \isasymin \ \isasymFF \ (fst\ y))\ \isasymand \ (\isasymexists W.\ (W\ \isasymin \ I)\ \isanewline
\ \ \ \ \ \ \ \ \ \ \ \ \ \ \ \ \ \isasymand \ (W\ \isasymsubseteq \ fst\ x\ \isasyminter \ fst\ y)\ \isasymand \ \isasymrho \ (fst\ x)\ W\ (snd\ x)\ \isanewline
\ \ \ \ \ \ \ \ \ \ \ \ \ \ \ \ \ \ \ \ \ \ \ \ \ \ \ \ \ \ \ \ \ \ \ \ \ \ \ \ \ \ \ \ \ \ \ \ \ =\ \isasymrho \ (fst\ y)\ W\ (snd\ y))"\isanewline
\isanewline
\isacommand{interpretation}\ rel:equivalence\ "(Sigma\ I\ \isasymFF )"\ "\{(x,\ y).\ x\ \isasymsim \ y\}"\isanewline
\isanewline
\isacommand{definition}\ class\_of::\ "'a\ set\ \isasymRightarrow \ 'b\ \isasymRightarrow \ ('a\ set\ \isasymtimes \ 'b)\ set"\ ("\isasymlfloor (\_,/\ \_)\isasymrfloor ")\isanewline
\ \ \isakeyword{where}\ "\isasymlfloor U,s\isasymrfloor \ \isasymequiv \ rel.Class\ (U,\ s)"\isanewline
\isanewline
\isacommand{definition}\ carrier\_direct\_lim::\ "('a\ set\ \isasymtimes \ 'b)\ set\ set"\isanewline
\ \ \isakeyword{where}\ "carrier\_direct\_lim\ \isasymequiv \ rel.Partition"\isanewline
\isanewline
\isacommand{definition}\ get\_lower\_bound::\ "'a\ set\ \isasymRightarrow \ 'a\ set\ \isasymRightarrow \ 'a\ set"\ \isakeyword{where}\isanewline
\ \ "get\_lower\_bound\ U\ V=\ (SOME\ W.\ W\ \isasymin \ I\ \isasymand \ W\ \isasymsubseteq \ U\ \isasymand \ W\ \isasymsubseteq \ V)"\isanewline
\isanewline
\isacommand{definition}\ add\_rel\ ::\ "('a\ set\ \isasymtimes \ 'b)\ set\ \isasymRightarrow \ ('a\ set\ \isasymtimes \ 'b)\ set\ \isanewline
\ \ \ \ \ \ \ \ \ \ \ \ \ \ \ \ \ \ \ \ \ \ \ \ \ \ \ \ \ \ \ \ \ \ \ \ \ \ \ \ \ \ \ \ \ \ \ \ \isasymRightarrow \ ('a\ set\ \isasymtimes \ 'b)\ set"\isanewline
\ \ \isakeyword{where}\ "add\_rel\ X\ Y\ \isasymequiv \ let\isanewline
\ \ \ \ \ \ \ \ \ \ \ \ \ \ x\ =\ (SOME\ x.\ x\ \isasymin \ X);\isanewline
\ \ \ \ \ \ \ \ \ \ \ \ \ \ y\ =\ (SOME\ y.\ y\ \isasymin \ Y);\isanewline
\ \ \ \ \ \ \ \ \ \ \ \ \ \ w\ =\ get\_lower\_bound\ (fst\ x)\ (fst\ y)\isanewline
\ \ \ \ \ \ \ \ \ \ \ \ in\isanewline
\ \ \ \ \ \ \ \ \ \ \ \ \ \ \isasymlfloor w,\ add\_str\ w\ (\isasymrho \ (fst\ x)\ w\ (snd\ x))\ (\isasymrho \ (fst\ y)\ w\ (snd\ y))\isasymrfloor "\isanewline
\isanewline
\isacommand{definition}\ mult\_rel\ ::\ "('a\ set\ \isasymtimes \ 'b)\ set\ \isasymRightarrow \ ('a\ set\ \isasymtimes \ 'b)\ set\ \isanewline
\ \ \ \ \ \ \ \ \ \ \ \ \ \ \ \ \ \ \ \ \ \ \ \ \ \ \ \ \ \ \ \ \ \ \ \ \ \ \ \ \ \ \ \ \ \ \ \ \ \ \ \ \isasymRightarrow \ ('a\ set\ \isasymtimes \ 'b)\ set"\isanewline
\ \ \isakeyword{where}\ "mult\_rel\ X\ Y\ \isasymequiv \ let\isanewline
\ \ \ \ \ \ \ \ \ \ \ \ \ \ x\ =\ (SOME\ x.\ x\ \isasymin \ X);\isanewline
\ \ \ \ \ \ \ \ \ \ \ \ \ \ y\ =\ (SOME\ y.\ y\ \isasymin \ Y);\isanewline
\ \ \ \ \ \ \ \ \ \ \ \ \ \ w\ =\ get\_lower\_bound\ (fst\ x)\ (fst\ y)\isanewline
\ \ \ \ \ \ \ \ \ \ \ \ in\isanewline
\ \ \ \ \ \ \ \ \ \ \ \ \ \ \isasymlfloor w,\ mult\_str\ w\ (\isasymrho \ (fst\ x)\ w\ (snd\ x))\ (\isasymrho \ (fst\ y)\ w\ (snd\ y))\isasymrfloor "\isanewline
\isanewline
\isacommand{lemma}\ direct\_{lim}\_{is}\_{ring}:\isanewline
\ \ \isakeyword{assumes}\ "U\ \isasymin \ I"\isanewline
\ \ \isakeyword{shows}\ "ring\ carrier\_direct\_lim\ add\_rel\ mult\_rel\ \isasymlfloor U,\ \isasymzero \isactrlbsub U\isactrlesub \isasymrfloor \ \isasymlfloor U,\ \isasymone \isactrlbsub U\isactrlesub \isasymrfloor "\isanewline
\isanewline
\isacommand{end}
\end{isabelle}

For every $U \in I$, there is a canonical inclusion from $\mathscr{F}(U)$ to $\varinjlim \mathscr{F}(U)$. 

% snippet of code to be inserted here for `definition canonical_fun` of Comm_Ring_Theory.thy
\begin{isabelle}
\isacommand{definition}\ (\isakeyword{in}\ direct\_lim)\ \isanewline
\ \ \ \ \ \ \ \ canonical\_fun\ ::\ "'a\ set\ \isasymRightarrow \ 'b\ \isasymRightarrow \ ('a\ set\ \isasymtimes \ 'b)\ set"\isanewline
\ \ \isakeyword{where}\ "canonical\_fun\ U\ x\ =\ \isasymlfloor U,\ x\isasymrfloor "
\end{isabelle}

The direct limit satisfies a useful universal property. Indeed, for every ring $A$ equipped with ring morphisms $\psi_U: \mathscr{F}(U) \rightarrow A$ satisfying $\psi_V \circ \rho_{UV} = \psi_U$ for every inclusion $V \subseteq U$, there exists a unique ring morphism $u$ making the following diagram commutes.
\[
\begin{tikzcd}
\mathscr{F}(U)  \arrow[rr, "\rho_{UV}"] \arrow[rd] \arrow[rdd, bend right, "\psi_U"'] & & \mathscr{F}(V) \arrow[ld]	\arrow[ldd, bend left, "\psi_V"]	\\
						& \varinjlim \mathscr{F}(U) \arrow[d, "u", dashed] &	\\
						& A &  
\end{tikzcd}
\] 
The counterpart in Isabelle is as follows.

% snippet of code to be inserted here for `proposition (in direct_lim) universal_property`  
\begin{isabelle}
\isacommand{proposition}\ (\isakeyword{in}\ direct\_lim)\ universal\_property:\isanewline
\ \ \isakeyword{fixes}\ A::\ "'c\ set"\ \isakeyword{and}\ \isasympsi ::\ "'a\ set\ \isasymRightarrow \ ('b\ \isasymRightarrow \ 'c)"\ \isanewline
\ \ \ \ \isakeyword{and}\ add::\ "'c\ \isasymRightarrow \ 'c\ \isasymRightarrow \ 'c"\ \isakeyword{and}\ mult::\ "'c\ \isasymRightarrow \ 'c\ \isasymRightarrow \ 'c"\ \isanewline
\ \ \ \ \isakeyword{and}\ zero::\ "'c"\ \isakeyword{and}\ one::\ "'c"\isanewline
\ \ \isakeyword{assumes}\ "ring\ A\ add\ mult\ zero\ one"\isanewline
\ \ \ \ \isakeyword{and}\ "\isasymAnd U.\ U\ \isasymin \ I\ \isasymLongrightarrow \ ring\_homomorphism\ (\isasympsi \ U)\ (\isasymFF \ U)\ (+\isactrlbsub U\isactrlesub )\ \isanewline
\ \ \ \ \ \ \ \ \ \ \ \ \ \ \ \ \ \ \ \ \ \ \ \ \ \ \ \ \ \ \ \ \ \ (\isasymcdot \isactrlbsub U\isactrlesub )\ \isasymzero \isactrlbsub U\isactrlesub \ \isasymone \isactrlbsub U\isactrlesub \ A\ add\ mult\ zero\ one"\isanewline
\ \ \ \ \isakeyword{and}\ "\isasymAnd U\ V\ x.\ \isasymlbrakk U\ \isasymin \ I;\ V\ \isasymin \ I;\ V\ \isasymsubseteq \ U;\ x\ \isasymin \ (\isasymFF \ U)\isasymrbrakk \ \isanewline
\ \ \ \ \ \ \ \ \ \ \ \ \ \ \ \ \ \ \ \ \ \ \ \ \ \ \ \ \ \ \ \ \ \ \ \ \ \ \ \ \ \ \isasymLongrightarrow \ (\isasympsi \ V\ \isasymcirc \ \isasymrho \ U\ V)\ x\ =\ \isasympsi \ U\ x"\isanewline
\ \ \isakeyword{shows}\ "\isasymforall V\isasymin I.\ \isasymexists !u.\ ring\_homomorphism\ u\ carrier\_direct\_lim\ add\_rel\ \isanewline
\ \ \ \ \ \ \ \ \ \ \ \ \ \ \ \ \ \ \ \ \ \ \ \ \ mult\_rel\ \isasymlfloor V,\isasymzero \isactrlbsub V\isactrlesub \isasymrfloor \ \isasymlfloor V,\isasymone \isactrlbsub V\isactrlesub \isasymrfloor \ A\ add\ mult\ zero\ one\isanewline
\ \ \ \ \ \ \ \ \ \ \ \ \ \ \ \ \ \ \ \isasymand \ (\isasymforall U\isasymin I.\ \isasymforall x\isasymin (\isasymFF \ U).\ (u\ \isasymcirc \ canonical\_fun\ U)\ x\ =\ \isasympsi \ U\ x)"
\end{isabelle}

One can instantiate direct limits in the following particular case.

\begin{definition}[stalks of a presheaf]
	Let $X$ be a topological space, $\mathscr{F}$ a presheaf of rings on $X$ and $x$ an element of $X$. \\
	The stalk $\mathscr{F}_x$ of $\mathscr{F}$ at $x$ is the direct limit over the set of all the open neighborhoods of $x$.   
\end{definition}

% snippet of code to be inserted here for `locale stalk` and the definitions `carrier_stalk`, `add_stalk`, `mult_stalk`, `zero_stalk`, `one_stalk`

\begin{isabelle}
\isacommand{locale}\ stalk\ =\ direct\_lim\ +\isanewline
\ \ \isakeyword{fixes}\ x::\ "'a"\isanewline
\ \ \isakeyword{assumes}\ is\_elem:\ "x\ \isasymin \ S"\ \isakeyword{and}\ index:\ "I\ =\ \{U.\ is\_open\ U\ \isasymand \ x\ \isasymin \ U\}"\isanewline
\isakeyword{begin}\isanewline
\isanewline
\isacommand{definition}\ carrier\_stalk::\ "('a\ set\ \isasymtimes \ 'b)\ set\ set"\isanewline
\ \ \isakeyword{where}\ "carrier\_stalk\ \isasymequiv \ dlim\ \isasymFF \ \isasymrho \ (neighborhoods\ x)"\isanewline
\isanewline
\isacommand{definition}\ add\_stalk\isanewline
\ \ \ \ ::\ "('a\ set\ \isasymtimes \ 'b)\ set\ \isasymRightarrow \ ('a\ set\ \isasymtimes \ 'b)\ set\ \isasymRightarrow \ ('a\ set\ \isasymtimes \ 'b)\ set"\isanewline
\ \ \isakeyword{where}\ "add\_stalk\ \isasymequiv \ add\_rel"\isanewline
\isanewline
\isacommand{definition}\ mult\_stalk\isanewline
\ \ \ \ ::\ "('a\ set\ \isasymtimes \ 'b)\ set\ \isasymRightarrow \ ('a\ set\ \isasymtimes \ 'b)\ set\ \isasymRightarrow \ ('a\ set\ \isasymtimes \ 'b)\ set"\isanewline
\ \ \isakeyword{where}\ "mult\_stalk\ \isasymequiv \ mult\_rel"\isanewline
\isanewline
\isacommand{definition}\ zero\_stalk::\ "'a\ set\ \isasymRightarrow \ ('a\ set\ \isasymtimes \ 'b)\ set"\isanewline
\ \ \isakeyword{where}\ "zero\_stalk\ V\ \isasymequiv \ class\_of\ V\ \isasymzero \isactrlbsub V\isactrlesub "\isanewline
\isanewline
\isacommand{definition}\ one\_stalk::\ "'a\ set\ \isasymRightarrow \ ('a\ set\ \isasymtimes \ 'b)\ set"\isanewline
\ \ \isakeyword{where}\ "one\_stalk\ V\ \isasymequiv \ class\_of\ V\ \isasymone \isactrlbsub V\isactrlesub "\isanewline
\isanewline
\isacommand{end}
\end{isabelle}

One further step towards schemes consists in introducing local rings, but first one needs to teach Isabelle simple facts about \emph{maximal ideals} starting with their definition. 

\begin{definition}[maximal ideal]
	Let $\mathfrak{m}$ be an ideal of $R$. The ideal $\mathfrak{m}$ is maximal if $\mathfrak{m} \neq R$ and there is no ideal $\mathfrak{a} \neq R$ such that $\mathfrak{m} \subset \mathfrak{a}$.
\end{definition}

% snippet of code to be inserted here for `locale max_ideal`
\begin{isabelle}
\isacommand{locale}\ max\_ideal\ =\ comm\_ring\ R\ "(+)"\ "(\isasymcdot )"\ "\isasymzero "\ "\isasymone "\ \isanewline
\ \ \ \ \ \ \ \ \ \ \ \ \ \ \ \ \ \ \ \ \ \ \ \ \ \ \ \ \ \ \ \ \ \ +\ ideal\ I\ \ R\ "(+)"\ "(\isasymcdot )"\ "\isasymzero "\ "\isasymone "\isanewline
\ \ \isakeyword{for}\ R\ \isakeyword{and}\ I\ \isakeyword{and}\ addition\ (\isakeyword{infixl}\ "+"\ 65)\ \isanewline
\ \ \ \ \isakeyword{and}\ multiplication\ (\isakeyword{infixl}\ "\isasymcdot "\ 70)\ \isakeyword{and}\ zero\ ("\isasymzero ")\ \isakeyword{and}\ unit\ ("\isasymone ")\ +\isanewline
\ \ \isakeyword{assumes}\ neq\_ring:\ "I\ \isasymnoteq \ R"\ \isanewline
\ \ \ \ \ \ \ \isakeyword{and}\ is\_max:\ "\isasymAnd \isasymaa .\ \isasymlbrakk ideal\ \isasymaa \ R\ (+)\ (\isasymcdot )\ \isasymzero \ \isasymone ;\isasymaa \ \isasymnoteq \ R;I\ \isasymsubseteq \ \isasymaa \ \isasymrbrakk \isasymLongrightarrow \ I\ =\ \isasymaa "
\end{isabelle}


Classic facts about maximal ideals include the fact that every maximal ideal is prime.

% snippet of code to be included for `proposition is_pr_ideal` of Comm_Ring_Theory.thy
\begin{isabelle}
\isacommand{proposition}\ (\isakeyword{in}\ max\_ideal)\ is\_pr\_ideal:\ "pr\_ideal\ R\ I\ (+)\ (\isasymcdot )\ \isasymzero \ \isasymone "
\end{isabelle}
	
We then introduce the notion of a \emph{local ring}.

\begin{definition}[local ring]
	A ring is a local ring if it is commutative and it has a unique maximal ideal. 	
\end{definition}

% snippet of code to be included here for `locale local_ring`
\begin{isabelle}
\isacommand{locale}\ local\_ring\ =\ ring\ +\isanewline
\ \ \isakeyword{assumes}\ is\_unique:\ "\isasymAnd I\ J.\ max\_lideal\ I\ R\ (+)\ (\isasymcdot )\ \isasymzero \ \isasymone \isanewline
\ \ \ \ \ \ \ \ \ \ \ \ \ \ \ \ \ \ \ \ \ \ \ \ \ \ \ \ \ \isasymLongrightarrow \ max\_lideal\ J\ R\ (+)\ (\isasymcdot )\ \isasymzero \ \isasymone \ \isasymLongrightarrow \ I\ =\ J"\isanewline
\ \ \ \ \isakeyword{and}\ has\_max\_lideal:\ "\isasymexists \isasymww .\ max\_lideal\ \isasymww \ R\ (+)\ (\isasymcdot )\ \isasymzero \ \isasymone "
\end{isabelle}

In particular, we prove as expected that the local ring of $R$ at $\mathfrak{p}$, previously denoted $R_{\mathfrak{p}}$, is a local ring.

% snippet of code to be included for `lemma local_ring_at_is_local` of Comm_Ring_Theory.thy.
\begin{isabelle}
\isacommand{lemma}\ (\isakeyword{in}\ pr\_ideal)\ local\_ring\_at\_is\_local:\isanewline
\ \ \isakeyword{shows}\ "local\_ring\ carrier\_local\_ring\_at\ add\_local\_ring\_at\ \isanewline
\ \ \ \ \ \ \ \ \ \ \ \ \ \ mult\_local\_ring\_at\ zero\_local\_ring\_at\ one\_local\_ring\_at"
\end{isabelle}

Local rings have their corresponding notion of morphisms. 

\begin{definition}[local homomorphism]
	Let $A$ and $B$ be two local rings and $\mathfrak{m}_A$, $\mathfrak{m}_B$ their respective maximal ideals. \\
	A local homomorphism $f: A \rightarrow B$ is a morphism of rings such that $f^{-1} (\mathfrak{m}_B) = \mathfrak{m}_A$. 
\end{definition}

% snippet of code to be inserted here `locale local_ring_morphism`	
\begin{isabelle}
\isacommand{locale}\ local\_ring\_morphism\ =\isanewline
\ \ \ \ \ \ \ \ \ \ source:\ local\_ring\ A\ "(+)"\ "(\isasymcdot )"\ \isasymzero \ \isasymone \ \isanewline
\ \ \ \ \ \ \ \ \ \ +\ target:\ local\_ring\ B\ "(+')"\ "(\isasymcdot ')"\ "\isasymzero '"\ "\isasymone '"\isanewline
\ \ \ \ \ \ \ \ \ \ +\ ring\_homomorphism\ f\ A\ "(+)"\ "(\isasymcdot )"\ "\isasymzero "\ "\isasymone "\ B\ "(+')"\ "(\isasymcdot ')"\ "\isasymzero '"\ "\isasymone '"\isanewline
\ \ \isakeyword{for}\ f\ \isakeyword{and}\ A\ \isakeyword{and}\ addition\ (\isakeyword{infixl}\ "+"\ 65)\ \isakeyword{and}\ multiplication\ (\isakeyword{infixl}\ "\isasymcdot "\ 70)\ \isanewline
\ \ \ \ \ \isakeyword{and}\ zero\ ("\isasymzero ")\ \isakeyword{and}\ unit\ ("\isasymone ")\ \isakeyword{and}\ B\ \isakeyword{and}\ addition'\ (\isakeyword{infixl}\ "+'{\kern0pt}'"\ 65)\ \isanewline
\ \ \ \ \ \isakeyword{and}\ multiplication'\ (\isakeyword{infixl}\ "\isasymcdot '{\kern0pt}'"\ 70)\ \isakeyword{and}\ zero'\ ("\isasymzero '{\kern0pt}'")\ \isakeyword{and}\ unit'\ ("\isasymone '{\kern0pt}'")\isanewline
\ \ +\ \isakeyword{assumes}\ preimage\_of\_max\_lideal:\isanewline
\ \ \ \ \ \ "\isasymAnd \isasymww \isactrlsub A\ \isasymww \isactrlsub B.\ max\_lideal\ \isasymww \isactrlsub A\ A\ (+)\ (\isasymcdot )\ \isasymzero \ \isasymone \ \isasymLongrightarrow \ max\_lideal\ \isasymww \isactrlsub B\ B\ (+')\ (\isasymcdot ')\ \isasymzero '\ \isasymone '\isanewline
\ \ \ \ \ \ \ \ \ \ \ \isasymLongrightarrow \ (f\isactrlsup \isasyminverse \ A\ \isasymww \isactrlsub B)\ =\ \isasymww \isactrlsub A"
\end{isabelle}

Finally, we are able to introduce \emph{locally ringed spaces}.

\begin{definition}[locally ringed space]
	A locally ringed space is a ringed space $(X, \mathscr{O}_X)$ such that the stalk $\mathscr{O}_{X, x}$ at $x$ is a local ring for every $x \in X$.
\end{definition}

% snippet of code to be introduced here for `locale locally_ringed_space`
\begin{isabelle}
\isacommand{locale}\ locally\_ringed\_space\ =\ ringed\_space\ +\isanewline
\ \ \isakeyword{assumes}\ stalks\_are\_local:\ "\isasymAnd x\ U.\ x\ \isasymin \ U\ \isasymLongrightarrow \ is\_open\ U\ \isasymLongrightarrow \ stalk.is\_local\ \isanewline
\ \ \ \ \ \ \ \ \ \ \ is\_open\ \isasymFF \ \isasymrho \ add\_str\ mult\_str\ zero\_str\ one\_str\ (neighborhoods\ x)\ x\ U"
\end{isabelle}

Moreover, one has the following result.

\begin{proposition}
	\label{prop:keyprop}
	For every $\mathfrak{p} \in \text{Spec}\,R$, the stalk $\mathscr{O}_{\spec, \mathfrak{p}}$ is isomorphic as a ring to $R_{\mathfrak{p}}$.	
\end{proposition}	

% snippet of code to be inserted here for `lemma stalk_at_prime_is_iso_to_local_ring_at_prime`
\begin{isabelle}
\isacommand{locale}\ key\_map\ =\ comm\_ring\ +\isanewline
\ \ \isakeyword{fixes}\ \isasympp ::\ "'a\ set"\ \isakeyword{assumes}\ is\_prime:\ "\isasympp \ \isasymin \ Spec"\isanewline
\isakeyword{begin}\isanewline
\isanewline
\isacommand{interpretation}\ pi:pr\_ideal\ R\ \isasympp \ "(+)"\ "(\isasymcdot )"\ \isasymzero \ \isasymone 
\isanewline
\isanewline
\isacommand{interpretation}\ st:\ stalk\ "Spec"\ is\_zariski\_open\ sheaf\_spec\ sheaf\_spec\_morphisms\isanewline
\ \ \ \ \ \ \ \ \ \ \ \ \ \ \ \ \isasymO b\ add\_sheaf\_spec\ mult\_sheaf\_spec\ zero\_sheaf\_spec\ one\_sheaf\_spec\isanewline
\ \ \ \ \ \ \ \ \ \ \ \ \ \ \ \ "\{U.\ is\_zariski\_open\ U\ \isasymand \ \isasympp \isasymin U\}"\ \isasympp \isanewline
\isanewline
\isacommand{lemma}\ stalk\_at\_prime\_is\_iso\_to\_local\_ring\_at\_prime:\isanewline
\ \ \isakeyword{assumes}\ "is\_zariski\_open\ V"\ \isakeyword{and}\ "\isasympp \ \isasymin \ V"\isanewline
\ \ \isakeyword{shows}\ "\isasymexists \isasymphi .\ ring\_isomorphism\ \isasymphi \ st.carrier\_stalk\ st.add\_stalk\ st.mult\_stalk\ \isanewline
\ \ \ \ \ \ \ \ \ \ \ \ \ \ (st.zero\_stalk\ V)\ (st.one\_stalk\ V)\ (R\ \isactrlbsub \isasympp \ (+)\ (\isasymcdot )\ \isasymzero \isactrlesub )\ \isanewline
\ \ \ \ \ \ \ \ \ \ \ \ \ \ (pi.add\_local\_ring\_at)\ (pi.mult\_local\_ring\_at)\ \isanewline
\ \ \ \ \ \ \ \ \ \ \ \ \ \ (pi.zero\_local\_ring\_at)\ (pi.one\_local\_ring\_at)"\isanewline
\isanewline
\isacommand{end}
\end{isabelle}

Now, it suffices to teach Isabelle that a ring which is isomorphic to a local ring is local.

% snippet of code to be inserted here for the statement of `lemma isomorphic_to_local_is_local`

\begin{isabelle}
\isacommand{lemma}\ isomorphic\_to\_local\_is\_local:\isanewline
\ \ \isakeyword{assumes}\ "local\_ring\ B\ addB\ multB\ zeroB\ oneB"\isanewline
\ \ \ \ \isakeyword{and}\ "ring\_isomorphism\ f\ A\ addA\ multA\ zeroA\ oneA\ B\ addB\ multB\ zeroB\ oneB"\isanewline
\ \ \isakeyword{shows}\ "local\_ring\ A\ addA\ multA\ zeroA\ oneA"
\end{isabelle}
	
\begin{cor}
	$(\text{Spec}\,R, \mathscr{O}_{\spec} )$ is a locally ringed space.
\end{cor}

As a sanity check for our formal definitions, we eventually prove in Isabelle that the spectrum of a commutative ring is indeed a locally ringed space, a result that required some amount of formal machinery, especially to prove the key proposition \ref{prop:keyprop}.

% snippet of code to be inserted here for the corresponding statement in Isabelle `lemma spec_is_locally_ringed_space`
\begin{isabelle}
\isacommand{lemma}\ (\isakeyword{in}\ comm\_ring)\ spec\_is\_locally\_ringed\_space:\isanewline
\ \ \isakeyword{shows}\ "locally\_ringed\_space\ Spec\ is\_zariski\_open\ sheaf\_spec\ \isanewline
\ \ \ \ \ \ \ \ \ \ \ \ \ \ sheaf\_spec\_morphisms\ \isasymO b\ add\_sheaf\_spec\ mult\_sheaf\_spec\ \isanewline
\ \ \ \ \ \ \ \ \ \ \ \ \ \ \ \ \ \ \ \ \ \ \ \ \ \ \ \ \ \ \ \ \ \ \ \ \ \ \ \ \ \ \ \ \ \ zero\_sheaf\_spec\ one\_sheaf\_spec"
\end{isabelle}

The required formal machinery and our formalization as a whole benefit from the interpretation mechanism that allows to keep statements and proof scripts concise. Without this mechanism the terms involved in the statements and proof scripts of such an intricate hierarchy of nested algebraic structures would have to be fed with many arguments, cluttering these statements and proof scripts to the point where they would become unreadable and difficult to use. Fortunately, the interpretation mechanism that comes with locales allows for concise and manageable terms with many implicit arguments which are declared beforehand using the interpretation mechanism and its commands \textit{interpretation} and \textit{interpret}.
  
Now, we introduce our last construction. \\
	Let $(X, \mathscr{O}_X)$, $(Y, \mathscr{O}_Y)$ be two ringed spaces and $(f, \phi_f)$ a morphism between them. Given $x \in X$, consider 
	\[
	I \coloneqq \lbrace V \mid V \, \text{is an open neighborhhood of}\, f(x) \rbrace .
	\]
	Since we have a morphism $\phi_f: \mathscr{O}_Y \rightarrow f_* \mathscr{O}_X $ of sheaves on $Y$, we get a map between the direct limits over $I$. 
	\[
	\mathscr{O}_{Y, f(x)} \rightarrow \displaystyle \varinjlim_I \mathscr{O}_X (f^{-1} V)
	\]
	Last, there is a natural inclusion from $\displaystyle \varinjlim_I \mathscr{O}_X (f^{-1} V)$ to $\mathscr{O}_{X, x}$, so eventually we get an induced morphism of rings from $\mathscr{O}_{Y, f(x)}$ to $\mathscr{O}_{X, x}$ which we will denote $\phi_{f, x}$. \\	
As usual we embed this construction into a dedicated locale in Isabelle.

% snippet of code to be inserted here for `locale ind_morphism_bwt_lim`	up to `definition induced_morphism`
\begin{isabelle}
\isacommand{locale}\ ind\_mor\_btw\_stalks\ =\ morphism\_ringed\_spaces\ +\isanewline
\ \ \isakeyword{fixes}\ x::"'a"\isanewline
\ \ \isakeyword{assumes}\ is\_elem:\ "x\ \isasymin \ X"\isanewline
\isakeyword{begin}\isanewline
\isanewline
\isacommand{interpretation}\ stx:stalk\ X\ is\_open\isactrlsub X\ \isasymO \isactrlsub X\ \isasymrho \isactrlsub X\ b\ add\_str\isactrlsub X\ mult\_str\isactrlsub X\ zero\_str\isactrlsub X\ \isanewline
\ \ one\_str\isactrlsub X\ "\{U.\ is\_open\isactrlsub X\ U\ \isasymand \ x\ \isasymin \ U\}"\ "x"\isanewline
\isacommand{proof}\ \isacommand{qed}\ (auto\ simp:\ is\_elem)
\isanewline
\isanewline
\isacommand{interpretation}\ stfx:\ stalk\ Y\ is\_open\isactrlsub Y\ \isasymO \isactrlsub Y\ \isasymrho \isactrlsub Y\ d\ add\_str\isactrlsub Y\ mult\_str\isactrlsub Y\ zero\_str\isactrlsub Y\ \isanewline
\ \ one\_str\isactrlsub Y\ "\{U.\ is\_open\isactrlsub Y\ U\ \isasymand \ (f\ x)\ \isasymin \ U\}"\ "f\ x"\isanewline
\isacommand{proof}\ \isacommand{qed}\ (auto\ simp:\ is\_elem)
\isanewline
\isanewline
\isacommand{definition}\ induced\_morphism::\ "('c\ set\ \isasymtimes \ 'd)\ set\ \isasymRightarrow \ ('a\ set\ \isasymtimes \ 'b)\ set"\ \isanewline
\ \ \isakeyword{where}\ "induced\_morphism\ \isasymequiv \ \isasymlambda C\ \isasymin \ stfx.carrier\_stalk.\ \ \isanewline
\ \ \ \ \ \ \ \ \ \ \ \ \ \ let\ r\ =\ (SOME\ r.\ r\ \isasymin \ C)\ in\ \isanewline
\ \ \ \ \ \ \ \ \ \ \ \ \ \ \ \ \ \ \ \ \ \ stx.class\_of\ (f\isactrlsup \isasyminverse \ X\ (fst\ r))\ (\isasymphi \isactrlsub f\ (fst\ r)\ (snd\ r))"\isanewline
\isanewline
\isacommand{end}
\end{isabelle}


This last construction is required to define morphisms between locally ringed spaces.
			
\begin{definition}[morphism of locally ringed spaces]
	A morphism of locally ringed spaces from $(X, \mathscr{O}_X)$ to $(Y, \mathscr{O}_Y)$ is a morphism $(f, \phi_f)$ between locally ringed spaces such that the induced map of local rings $\phi_{f, x}: \mathscr{O}_{Y, f(x)} \rightarrow \mathscr{O}_{X, x}$ is a local homomorphism for every $x \in X$.  
\end{definition}

% snippet of code to be introduced here for `locale morphism_locally_ringed_spaces`, still in Comm_Ring_Theory.thy
\begin{isabelle}
\isacommand{locale}\ morphism\_locally\_ringed\_spaces\ =\ morphism\_ringed\_spaces\ +\isanewline
\ \ \isakeyword{assumes}\ are\_local\_morphisms:\isanewline
\ \ \ \ "\isasymAnd x\ V.\ \isasymlbrakk x\ \isasymin \ X;\ is\_open\isactrlsub Y\ V;\ f\ x\ \isasymin \ V\isasymrbrakk \ \isasymLongrightarrow \ \isanewline
\ \ \ \ \ \ \ \ \ \ \ \ \ \ ind\_mor\_btw\_stalks.is\_local\ X\ is\_open\isactrlsub X\ \isasymO \isactrlsub X\ \isasymrho \isactrlsub X\ add\_str\isactrlsub X\ \isanewline
\ \ \ \ \ \ \ \ \ \ \ \ \ \ \ \ \ \ mult\_str\isactrlsub X\ zero\_str\isactrlsub X\ one\_str\isactrlsub X\ is\_open\isactrlsub Y\ \isasymO \isactrlsub Y\ \isasymrho \isactrlsub Y\ add\_str\isactrlsub Y\ \isanewline
\ \ \ \ \ \ \ \ \ \ \ \ \ \ \ \ \ \ \ \ \ \ mult\_str\isactrlsub Y\ zero\_str\isactrlsub Y\ one\_str\isactrlsub Y\ f\ x\ V\ \isasymphi \isactrlbsub X\ is\_open\isactrlsub X\ \isasymO \isactrlsub X\ \isanewline
\ \ \ \ \ \ \ \ \ \ \ \ \ \ \ \ \ \ \ \ \ \ \ \ \ \ \ \ \ \ \ \ \ \ \ \ \ \ \ \ \ \ \ \ \ \ \ \ \isasymrho \isactrlsub X\ is\_open\isactrlsub Y\ \isasymO \isactrlsub Y\ \isasymrho \isactrlsub Y\ f\ \isasymphi \isactrlsub f\ x\isactrlesub "
\end{isabelle}

\begin{remark}
	A morphism $(f, \phi_f)$ between locally ringed spaces is an isomorphism if and only if $f$ is an isomorphism between topological spaces, \textit{i.e.} a homeomorphism, and $\phi_f$ is an isomorphism between sheaves of rings.
\end{remark}

% snippet of code to be introduced for `iso_locally_ringed_spaces`
\begin{isabelle}
\isacommand{locale}\ iso\_locally\_ringed\_spaces\ =\ morphism\_locally\_ringed\_spaces\ +\isanewline
\ \ \isakeyword{assumes}\ is\_homeomorphism:\ "homeomorphism\ X\ is\_open\isactrlsub X\ Y\ is\_open\isactrlsub Y\ f"\ \isanewline
\ \ \ \ \ \ \isakeyword{and}\ is\_iso\_of\_sheaves:\ "iso\_sheaves\_of\_rings\ Y\ is\_open\isactrlsub Y\ \isasymO \isactrlsub Y\ \isasymrho \isactrlsub Y\ d\ \isanewline
\ \ \ \ \ \ \ \ \ \ \ \ add\_str\isactrlsub Y\ mult\_str\isactrlsub Y\ zero\_str\isactrlsub Y\ one\_str\isactrlsub Y\ im\_sheaf\ im\_sheaf\_morphisms\isanewline
\ \ \ \ \ \ \ \ \ \ \ \ \ \ \ b\ add\_im\_sheaf\ mult\_im\_sheaf\ zero\_im\_sheaf\ one\_im\_sheaf\ \isasymphi \isactrlsub f"
\end{isabelle}

Ultimately, we reach our goal: teaching schemes to Isabelle.	

\begin{definition}[affine scheme]
	An affine scheme is a locally ringed space $(X, \mathscr{O}_X)$ which is isomorphic (as a locally ringed space) to the spectrum $(\text{Spec}\,R, \mathscr{O}_{\spec} )$ for some commutative ring $R$. 
\end{definition}

% snippet of code to be introduced for `locale affine_scheme` in Scheme_Theory.thy
\begin{isabelle}
\isacommand{locale}\ affine\_scheme\ =\ comm\_ring\ \isanewline
\ \ \ \ +\ locally\_ringed\_space\ X\ is\_open\ \isasymO \isactrlsub X\ \isasymrho \ b\ add\_str\ mult\_str\ \isanewline
\ \ \ \ \ \ \ \ \ \ \ \ \ \ \ \ \ \ \ \ \ \ \ \ \ \ \ \ \ \ \ \ \ \ \ \ \ \ \ \ \ \ \ \ \ \ \ \ \ \ \ \ zero\_str\ one\_str\ \isanewline
\ \ \ \ +\ iso\_locally\_ringed\_spaces\ X\ is\_open\ \isasymO \isactrlsub X\ \isasymrho \ b\ add\_str\ mult\_str\isanewline
\ \ \ \ \ \ \ \ zero\_str\ one\_str\ "Spec"\ is\_zariski\_open\ sheaf\_spec\ \isanewline
\ \ \ \ \ \ \ \ \ \ \ \ sheaf\_spec\_morphisms\ \isasymO b\ "\isasymlambda U.\ add\_sheaf\_spec\ U"\isanewline
\ \ \ \ \ \ \ \ \ \ \ \ \ \ \ \ \ \ "\isasymlambda U.\ mult\_sheaf\_spec\ U"\ "\isasymlambda U.\ zero\_sheaf\_spec\ U"\ \isanewline
\ \ \ \ \ \ \ \ \ \ \ \ \ \ \ \ \ \ \ \ \ \ \ \ \ \ \ \ \ \ \ \ \ \ \ \ \ \ \ \ \ \ "\isasymlambda U.\ one\_sheaf\_spec\ U"\ f\ \isasymphi \isactrlsub f\isanewline
\ \ \ \ \ \ \ \ \ \ \isakeyword{for}\ X\ is\_open\ \isasymO \isactrlsub X\ \isasymrho \ b\ add\_str\ mult\_str\ zero\_str\ one\_str\ f\ \isasymphi \isactrlsub f
\end{isabelle}

Of course, spectra of commutative rings being locally ringed spaces provide a class of affine schemes.

% snippet of code to be introduced here for `lemma spec_is_affine_scheme`
\begin{isabelle}
\isacommand{lemma}\ (\isakeyword{in}\ comm\_ring)\ spec\_is\_affine\_scheme:\isanewline
\ \ \isakeyword{shows}\ "affine\_scheme\ R\ (+)\ (\isasymcdot )\ \isasymzero \ \isasymone \ Spec\ is\_zariski\_open\ sheaf\_spec\ \isanewline
\ \ \ \ \ \ \ \ \ \ \ \ sheaf\_spec\_morphisms\ \isasymO b\ (\isasymlambda U.\ add\_sheaf\_spec\ U)\ \isanewline
\ \ \ \ \ \ \ \ \ \ \ \ \ \ (\isasymlambda U.\ mult\_sheaf\_spec\ U)\ (\isasymlambda U.\ zero\_sheaf\_spec\ U)\ \isanewline
\ \ \ \ \ \ \ \ \ \ \ \ \ \ \ \ \ \ \ \ \ \ \ \ (\isasymlambda U.\ one\_sheaf\_spec\ U)\ (identity\ Spec)\ \isanewline
\ \ \ \ \ \ \ \ \ \ \ \ \ \ \ \ \ \ \ \ \ \ \ \ \ \ \ \ \ \ \ \ \ \ \ \ \ \ \ \ \ \ \ \ \ \ (\isasymlambda U.\ identity\ (\isasymO \ U))"
\end{isabelle}

\begin{definition}[scheme]
	A scheme is a locally ringed space $(X, \mathscr{O}_X)$ in which every point has an open neighborhood $U$ such that the topological space $U$, together with the sheaf $\mathscr{O}_X | _U$, is an affine scheme.
\end{definition}

% snippet of code to be inserted here for `locale scheme`
\begin{isabelle}
\isacommand{locale}\ scheme\ =\ comm\_ring\ \isanewline
\ \ \ \ \ +\ locally\_ringed\_space\ X\ is\_open\ \isasymO \isactrlsub X\ \isasymrho \ b\ add\_str\ mult\_str\isanewline
\ \ \ \ \ \ \ \ \ \ \ \ \ \ \ \ \ \ \ \ \ \ \ \ \ \ \ \ \ \ \ \ \ \ \ \ \ \ \ \ \ \ \ \ \ \ \ \ \ \ \ zero\_str\ one\_str\ \isanewline
\ \ \isakeyword{for}\ X\ is\_open\ \isasymO \isactrlsub X\ \isasymrho \ b\ add\_str\ mult\_str\ zero\_str\ one\_str\ +\isanewline
\ \ \isakeyword{assumes}\ are\_affine\_schemes:\ "\isasymAnd x.\ x\ \isasymin \ X\ \isasymLongrightarrow \ (\isasymexists U.\ x\isasymin U\ \isanewline
\ \ \ \ \ \ \ \ \ \ \isasymand \ is\_open\ U\ \isasymand \ \ (\isasymexists f\ \isasymphi \isactrlsub f.\ affine\_scheme\ \ R\ (+)\ (\isasymcdot )\ \isasymzero \ \isasymone \ U\ \isanewline
\ \ \ \ \ \ \ \ \ \ \ (ind\_topology.ind\_is\_open\ X\ is\_open\ U)\isanewline
\ \ \ \ \ \ \ \ \ \ \ \ \ \ (ind\_sheaf.ind\_sheaf\ \isasymO \isactrlsub X\ U)\ \isanewline
\ \ \ \ \ \ \ \ \ \ \ \ \ \ \ \ \ \ (ind\_sheaf.ind\_ring\_morphisms\ \isasymrho \ U)\ b\ \isanewline
\ \ \ \ \ \ \ \ \ \ \ \ \ \ \ \ \ \ \ \ \ (ind\_sheaf.ind\_add\_str\ add\_str\ U)\ \isanewline
\ \ \ \ \ \ \ \ \ \ \ \ \ \ \ \ \ \ \ \ \ \ \ \ (ind\_sheaf.ind\_mult\_str\ mult\_str\ U)\isanewline
\ \ \ \ \ \ \ \ \ \ \ \ \ \ \ \ \ \ \ \ \ \ \ \ \ \ \ (ind\_sheaf.ind\_zero\_str\ zero\_str\ U)\ \isanewline
\ \ \ \ \ \ \ \ \ \ \ \ \ \ \ \ \ \ \ \ \ \ \ \ \ \ \ \ \ \ (ind\_sheaf.ind\_one\_str\ one\_str\ U)\ f\ \isasymphi \isactrlsub f))"
\end{isabelle}
To catch on the state of the art in \cite{schemesinLean}, we finally prove in Isabelle that an affine scheme is a scheme and we give the example of the empty scheme $(\emptyset, \mathscr{O}_\emptyset)$, where $\mathscr{O}_\emptyset(\emptyset)$ is the zero ring $\lbrace 0 \rbrace$. 

% snippets of code to be inserted here for `lemma affine_scheme_is_scheme` and for `empty_scheme_is_scheme`
\begin{isabelle}
\isacommand{lemma}\ (\isakeyword{in}\ affine\_scheme)\ affine\_scheme\_is\_scheme:\isanewline
\ \ \isakeyword{shows}\ "scheme\ R\ (+)\ (\isasymcdot )\ \isasymzero \ \isasymone \ X\ is\_open\ \isasymO \isactrlsub X\ \isasymrho \ b\ add\_str\ \isanewline
\ \ \ \ \ \ \ \ \ \ \ \ \ \ \ \ \ \ \ \ \ \ \ \ \ \ \ \ \ \ \ \ \ \ \ \ \ \ \ \ \ \ \ \ \ mult\_str\ zero\_str\ one\_str"
\end{isabelle}

\begin{isabelle}
\isacommand{lemma}\ empty\_scheme\_is\_scheme:\isanewline
\ \ \isakeyword{shows}\ "scheme\ \{0::nat\}\ (\isasymlambda x\ y.\ 0)\ (\isasymlambda x\ y.\ 0)\ 0\ 0\ \{\}\ (\isasymlambda U.\ U=\{\})\ \isanewline
\ \ \ \ \ \ \ \ \ \ \ \ \ \ (\isasymlambda U.\ \{0\})\ (\isasymlambda U\ V.\ identity\{0::nat\})\ 0\ (\isasymlambda U\ x\ y.\ 0)\ \isanewline
\ \ \ \ \ \ \ \ \ \ \ \ \ \ \ \ \ \ \ \ \ \ \ \ \ \ \ \ \ \ \ \ \ \ \ \ \ \ \ \ \ \ \ (\isasymlambda U\ x\ y.\ 0)\ (\isasymlambda U.\ 0)\ (\isasymlambda U.\ 0)"
\end{isabelle}

\section{Concluding Thoughts}
\setlength\parindent {5ex}

For our formalization we built upon Ballarin's work \cite{ballarin-exploring} formalizing parts of Jacobson's \textit{Basic Algebra} textbook. We believe Ballarin's nice formalization could serve as a first stepping stone to a new algebra library for Isabelle. We also hope the present work could serve as a second stepping stone in that direction. Also, we seized the opportunity of formalizing schemes to build a new topology library despite the two existing formalizations of topology in Isabelle/HOL, namely \textit{HOL-Topological\_Spaces.thy}\footnote{\url{https://isabelle.in.tum.de/dist/library/HOL/HOL/Topological_Spaces.html}} and \textit{HOL-Analysis/Abstract\_Topology}\footnote{\url{https://isabelle.in.tum.de/dist/library/HOL/HOL-Analysis/Abstract_Topology.html}}. Our new topology library is entirely built on modern locales without using type classes or type declarations (via the \textit{typedef} command). In particular, our topological spaces have explicit carrier sets as part of their data instead of using \textit{UNIV} or having to define it later as the union of all the open sets. 
% snippet of code to be included here for `locale topological_space` of Topological_Space_Theory.thy
\begin{isabelle}
\isacommand{locale}\ topological\_space\ =\ \isanewline
\ \ \isakeyword{fixes}\ S\ ::\ "'a\ set"\ \isakeyword{and}\ is\_open\ ::\ "'a\ set\ \isasymRightarrow \ bool"\isanewline
\ \ \isakeyword{assumes}\ open\_space\ [simp,\ intro]:\ "is\_open\ S"\ \isanewline
\ \ \ \ \isakeyword{and}\ open\_empty\ [simp,\ intro]:\ "is\_open\ \{\}"\ \isanewline
\ \ \ \ \isakeyword{and}\ open\_imp\_subset:\ "is\_open\ U\ \isasymLongrightarrow \ U\ \isasymsubseteq \ S"\isanewline
\ \ \ \ \isakeyword{and}\ open\_inter\ [intro]:\ "\isasymlbrakk is\_open\ U;\ is\_open\ V\isasymrbrakk \ \isasymLongrightarrow \ is\_open\ (U\ \isasyminter \ V)"\ \isanewline
\ \ \ \ \isakeyword{and}\ open\_union\ [intro]:\ \isanewline
\ \ \ \ \ \ \ \ \ \ \ \ \ \ \ \ "\isasymAnd F::('a\ set)\ set.\ (\isasymAnd x.\ x\ \isasymin \ F\ \isasymLongrightarrow \ is\_open\ x)\isanewline
\ \ \ \ \ \ \ \ \ \ \ \ \ \ \ \ \ \ \ \ \ \ \ \ \ \ \ \ \ \ \ \ \ \ \ \ \ \ \ \ \ \ \ \ \ \ \ \ \ \isasymLongrightarrow \ is\_open\ (\isasymUnion x\isasymin F.\ x)"
\end{isabelle}
Within the locale for topological spaces the axiom \textit{open\_imp\_subset} enforces that an open is automatically a subset of the ambient space. This is a welcome addition, since it avoids having to add this assumption in every lemma mentioning an open set among its assumptions, making the economy of dozens of trivial assumptions. Our library includes basic facts about topological spaces like the topology generated by a basis, covers, the induced topology on a subset and homeomorphisms. 
% snippets of code to be included here for `inductive generated_topology`, `lemma generated_topology_is_topology`  and `locale homeomorphism`
\begin{isabelle}
\isacommand{inductive}\ generated\_topology\ ::\ "'a\ set\ \isasymRightarrow \ 'a\ set\ set\ \isasymRightarrow \ 'a\ set\ \isasymRightarrow \ bool"\ \isanewline
\ \ \ \ \isakeyword{for}\ S\ ::\ "'a\ set"\ \isakeyword{and}\ B\ ::\ "'a\ set\ set"\isanewline
\ \ \isakeyword{where}\isanewline
\ \ \ \ UNIV:\ "generated\_topology\ S\ B\ S"\isanewline
\ \ |\ Int:\ "generated\_topology\ S\ B\ (U\ \isasyminter \ V)"\ \isanewline
\ \ \ \ \ \ \ \ \ \ \ \ \isakeyword{if}\ "generated\_topology\ S\ B\ U"\ \isakeyword{and}\ "generated\_topology\ S\ B\ V"\isanewline
\ \ |\ UN:\ "generated\_topology\ S\ B\ (\isasymUnion K)"\ \isanewline
\ \ \ \ \ \ \ \ \ \ \ \ \isakeyword{if}\ "(\isasymAnd U.\ U\ \isasymin \ K\ \isasymLongrightarrow \ generated\_topology\ S\ B\ U)"\isanewline
\ \ |\ Basis:\ "generated\_topology\ S\ B\ b"\ \isakeyword{if}\ "b\ \isasymin \ B\ \isasymand \ b\ \isasymsubseteq \ S"
\end{isabelle}
\begin{isabelle}
\isacommand{lemma}\ generated\_topology\_is\_topology:\isanewline
\ \ \isakeyword{fixes}\ S::\ "'a\ set"\ \isakeyword{and}\ B::\ "'a\ set\ set"\isanewline
\ \ \isakeyword{shows}\ "topological\_space\ S\ (generated\_topology\ S\ B)"
\end{isabelle}
\begin{isabelle}
\isacommand{locale}\ homeomorphism\ =\ \isanewline
\ \ continuous\_map\ +\ bijective\_map\ f\ S\ S'\ +\ \isanewline
\ \ continuous\_map\ S'\ is\_open'\ S\ is\_open\ "inverse\_map\ f\ S\ S'"
\end{isabelle}

Actually, this reliance on locales, avoiding type classes and type declarations whenever possible, pervades our development. It inaugurates a new discipline of formalizing mathematics with Isabelle/HOL which is exemplary in bridging the gap between set theory and type theory within Isabelle/HOL, while forgoing the extension of Isabelle's logic with a new rule for type definition as proposed by Kun\v{c}ar and Popescu in their types-to-sets framework \cite{typestosets}. \\
The formalized material has a size of roughly 7300 lines of code, including all the prerequisites in commutative  algebra (prime and maximal ideals, localization of a ring and much else) and even our partial reconstruction of the topology library (down to the notion of a topological space). Given the output of this work, a formalization of schemes in Isabelle, such a relatively short formalization conveys the power of Isabelle module systems. \\
To summarize, our work makes a triple contribution to Isabelle/HOL: a first stepping stone towards a new topology library, a second stepping stone towards a new algebra library and finally it develops a library for modern algebraic geometry. \\
Most importantly, our work meets the challenge of formalizing for the first time Grothendieck's schemes in simple type theory and in doing so it demonstrates that intricate dependencies of algebraic structures can be managed elegantly in Isabelle. We hope this work will help mathematicians interested in the formalization of mathematics to make a more informed decision about which systems are available to them.

Last, it seems that our experience in Isabelle to formalize schemes was much smoother than the corresponding one in Lean, since Kevin Buzzard reported on his blog the serious difficulties he faced during his formalization in Lean:
\begin{quote}
	The project is completely incompatible with modern Lean and mathlib, but if you compile it then you get a sorry-free proof that an affine scheme is a scheme. During our proof we ran into a huge problem because our blueprint, the Stacks Project, assumed that $R[1/f][1/g]=R[1/fg]$, and this turns out to be unprovable for Lean’s version of equality [\dots]. The Lean community wrestled with this idea, and has ultimately come up with a beefed-up notion of equality for which the identity is now true.\footnote{\url{https://xenaproject.wordpress.com/2020/06/05/the-sphere-eversion-project/}}
\end{quote}	
As we understand it, this difficulty basically comes from defining the sheaf structure of $\spec$ first on the basis given by the so-called \textit{basic open sets} before extending it to all open subsets. We did not meet this difficulty in Isabelle, since we did not follow this sheaf-on-a-basis approach and the sheaves in our library are always defined from the get-go on all open subsets instead. \\
Also, Buzzard \textit{et al.} mention another difficulty encountered when proving that an affine scheme is a scheme. 
\begin{quote}
	This means that rewriting the equality $\iota(U) = U$ can cause technical problems with data-carrying types such as $\mathscr{O}_X(\iota(U))$ which depends on $\iota(U)$ (although in our case they were surmountable). This is a technical issue with dependent type theory and can sometimes indicate that one is working with the wrong definitions. \cite[3.5]{schemesinLean}
\end{quote}	
where $\iota: \spec \rightarrow \spec$ is here simply the identity map. If, again, we gave a slightly different formulation for the definition of a scheme, following Hartshorne~\cite{hartshorne} instead of the Stacks project \cite{stacksproject} as they did, it seems that this difficulty cannot possibly have a direct counterpart in Isabelle, since it arose from the difference between the so-called \textit{equality types}, also known as \textit{identity types}, and the \textit{definitional equality}, a difference which is peculiar to dependent type theories. \\
If we were much more fortunate with our formalization in Isabelle, it is worth mentioning that this difficulty was eventually overcome in Lean using a trick.  
\begin{quote}
	Fortunately, in our case, Reid Barton pointed out the following extraordinary trick to us: we can define the map $\mathscr{O}_X(\iota(U)) \rightarrow \mathscr{O}_X(U)$ using restriction rather than trying to force it to be the identity! \cite[3.5.]{schemesinLean}
\end{quote}	    
However, all these difficulties in Lean, where three attempts at defining schemes were required, raise the question of whether dependent types could be, at least in some situations, more of a burden than an advantage \cite{PaulsonLamport}. Moreover, in the age of machine learning, one may expect there is a trade-off between the expressiveness of a type theory and one's ability to subject it to automation, for instance through SMT solvers and the so-called hammers embedded in some proof assistants. A dual question is to know how deep one can go formalizing mathematics in a simple type theory. This work provides a first hint at an answer that should not be ruled out: simple type theory is not too simple.        

\section*{Acknowledgments}

Work on this paper was supported by the ERC Advanced Grant ALEXANDRIA (Project GA 742178).  \\
Anthony Bordg thanks Kevin Buzzard for his stimulating wit and his communicative energy.			

\bibliographystyle{plain}
\bibliography{ALEXANDRIA}

\end{document}
