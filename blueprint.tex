\documentclass[12pt]{report}

\usepackage[utf8]{inputenc}
\usepackage[T1]{fontenc}
\usepackage{babel}

\usepackage{amscd, amssymb, mathtools, amsthm, amsmath, mathrsfs}
\usepackage{enumerate}
% Palatino for main text and math
\usepackage[osf,sc]{mathpazo}

% Helvetica for sans serif
% (scaled to match size of Palatino)
\usepackage[scaled=0.90]{helvet}

% Bera Mono for monospaced
% (scaled to match size of Palatino)
\usepackage[scaled=0.85]{beramono}

\usepackage{hyperref}

% To write appendixes
\usepackage[page]{appendix}


\renewcommand{\thesubsection}{\arabic{subsection}}
%
%
\newenvironment{eq}{\begin{equation}}{\end{equation}}
%
\newenvironment{remarks}{{\bf Remarks}:\begin{enumerate}}{\end{enumerate}}
\newenvironment{examples}{{\bf Examples}:\begin{enumerate}}{\end{enumerate}}  
%
\newtheorem{proposition}{Proposition}[section]
\newtheorem{lemma}[proposition]{Lemma}
\newtheorem{definition}[proposition]{Definition}
\newtheorem{theorem}[proposition]{Theorem}
\newtheorem{cor}[proposition]{Corollary}
\newtheorem{conjecture}{Conjecture}
\newtheorem{pretheorem}[proposition]{Pretheorem}
\newtheorem{hypothesis}[proposition]{Hypothesis}
\newtheorem{example}[proposition]{Example}
\newtheorem*{rem}{Remark}
\newtheorem{remark}[proposition]{Remark}
\newtheorem{ex}[proposition]{Exercise}
\newtheorem{cond}[proposition]{Conditions}
\newtheorem{cons}[proposition]{Construction}
%
\newtheorem{problem}[proposition]{Problem}
\newtheorem{construction}[proposition]{Construction}
%
\newtheorem*{nota}{Notation}
\newtheorem{notation}{Notation}
\newtheorem*{axiom}{Axiom}

% For diagrams:
\usepackage{tikz-cd}
\usetikzlibrary{cd}

% To avoid automatic identation of paragraphs:
\usepackage[parfill]{parskip}

% To create an index:
\usepackage{makeidx}

% To make the index and bibliography appear in the table of contents:
\usepackage[nottoc]{tocbibind}

\makeindex

\begin{document}
	
\title{Blueprint}
\author{}
\date{\today}
\maketitle


Let $k$ be an algebraically closed field and $A$ denote the ring $k \left[ x_1, \dots, x_n \right]$. \\

\begin{definition}[affine $n$-space] 
	The affine $n$-space over $k$, denoted $\mathbb{A}^n_k$ or simply $\mathbb{A}^n$, is the set of all $n$-tuples of elements of $k$.  
\end{definition}

\begin{remark}
	An element $P$ of $\mathbb{A}^n$ will be called a point. \\
	If $P = (a_1, \dots, a_n)$ with $a_i \in k$, then $a_i$ will be called the $i$th coordinate of $P$.
\end{remark}

\begin{notation}
	if $P = (a_1, \dots, a_n)$ (\textit{resp.} $f$) belongs to $\mathbb{A}^n$ (\textit{resp.} $A$), then $f(P)$ is $f(a_1, \dots, a_n)$.
\end{notation}		

\begin{definition}[zero set]
	If $T$ is any subset of $A$, one defines the zero set of $T$, denoted $Z(T)$, to be the common zeros of all the elements of $T$. \\
	In other words one has
	\[
	Z(T) \coloneqq \lbrace P \in \mathbb{A}^n \mid f(P) = 0 \quad \text{for all}\, f \in T \rbrace \, . 
	\]
\end{definition}

\begin{definition}[algebraic set]
	A subset $Y$ of $\mathbb{A}^n$ is an algebraic set if there exists a subset $T \subseteq A$ such that $Y = Z(T)$. 
\end{definition}

\begin{ex}
	Prove the empty space $\emptyset$ and the whole space $\mathbb{A}^n$ are algebraic sets.
\end{ex}

\begin{ex}
	Prove the intersection of any family of algebraic sets is an algebraic set.
\end{ex}

\begin{ex}
	Prove the union of two algebraic sets is an algebraic set.
\end{ex}		

\begin{definition}[Zariski topology]
	One defines the Zariski topology on $\mathbb{A}^n$ by taking the open sets to be the complements of the algebraic sets.
\end{definition}

\begin{ex}
	Prove the Zariski topology is a topology. 
\end{ex}

\begin{definition}[irreducible subset]
	A nonempty subset $Y$ of a topological space $X$ is irreducible if $Y$ cannot be expressed as the union of two proper closed subsets of $X$. 
\end{definition}

\begin{definition}[affine algebraic variety]
	An affine algebraic variety or simply an affine variety is an irreducible closed subset of $\mathbb{A}^n$ endowed with the induced topology. 
\end{definition}

\begin{definition}[quasi-affine variety]
	A quasi-affine variety is an open subset of an affine variety. 
\end{definition}

Let $R$ be a commutative ring. 

\begin{definition}[ideal]
	An ideal $\mathfrak{a}$ of $R$ is a subset of $R$ which is an additive subgroup such that $R\mathfrak{a} = \mathfrak{a}$.
\end{definition}

\begin{ex}
	Prove that $\lbrace 0 \rbrace$ and $R$ are ideals of $R$.
\end{ex}

\begin{ex}
	Let $\mathfrak{a}$ and $\mathfrak{b}$ be two ideals of $R$. Check that $\mathfrak{a} \mathfrak{b}$ is also an ideal of $R$.
\end{ex}		

\begin{definition}[zero divisor]
	Two elements $x$ and $y$ of $R$ are zero divisors if $x \neq 0$, $y \neq 0$ and $x y = 0$.
\end{definition}

\begin{definition}[entire ring]
	The ring $R$ is said to be entire if $1 \neq 0$ and there are no zero divisors in $R$. 
\end{definition}
		
\begin{definition}[prime ideal]
	A prime ideal $\mathfrak{p}$ of $R$ is an ideal $\mathfrak{p} \neq R$ such that $R/\mathfrak{p}$ is an entire ring. 	
\end{definition}

\begin{ex}
	Equivalently, a prime ideal is an ideal $\mathfrak{p} \neq R$ such that $x y \in \mathfrak{p}$ implies $x \in \mathfrak{p}$ or $y \in \mathfrak{p}$ for every elements $x$ and $y$ in $R$.
\end{ex}

\begin{remark}
	In a formal setting it is probably better to use the characterization above as the definition of prime ideals.
\end{remark}

\begin{notation}
	If $\mathfrak{i}$ is an ideal of $R$, $V(\mathfrak{a})$ denotes the set of all prime ideals of $R$ which contain $\mathfrak{a}$. 
\end{notation}

\begin{remark}
	On has $V(R) = \emptyset$ and $V(\lbrace 0 \rbrace)$ is the set of all prime ideals of $R$.
\end{remark}	

\begin{ex}
Prove the equality $V(\mathfrak{a} \mathfrak{b}) = V(\mathfrak{a}) \cup V(\mathfrak{b})$.	
\end{ex}

\begin{ex}
	Prove the equality $V(\sum \mathfrak{a}_i) = \bigcap V(\mathfrak{a}_i)$ for any familly of prime ideals.
\end{ex}

\begin{ex}
	Prove the subsets of the form $V(\mathfrak{a})$ as the closed subsets define a topology on the set of all primes ideals of $R$. 
\end{ex}

\begin{notation}
	Spec $R$ will denote the topological space of all prime ideals of $R$.
\end{notation}

\begin{definition}[presheaf of rings]
	Let $X$ be a topological space. A presheaf $\mathscr{F}$ of rings on $X$ consists of the following data:
	\begin{itemize}
		\item for every open set $U$, a ring $\mathscr{F}(U)$
		\item for every inclusion $V \subseteq U$ of open subsets, a morphism of rings $\rho_{UV}: \mathscr{F}(U) \rightarrow \mathscr{F}(V)$  
	\end{itemize}
satisfying 
	\begin{enumerate}
		\item $\mathscr{F}(\emptyset) = \lbrace 0 \rbrace$
		\item $\rho_{UU}$ is the identity map for every open subset $U$
		\item  If $W \subseteq V \subseteq U$ are three open subsets, then $\rho_{UW} = \rho_{VW} \circ \rho_{UV}$.
	\end{enumerate}
\end{definition}

\begin{notation}
	The elements of $\mathscr{F}(U)$ are sometimes called the sections of the presheaf $\mathscr{F}$ and given $s \in \mathscr{F}(U)$, $s\restriction V$ denotes the element $\rho_{UV}(s)$.
\end{notation}	

\begin{definition}[sheaf of rings]
	A sheaf of rings on a topological space $X$ is a presheaf on $X$ that satisfies the following additional properties:
	\begin{enumerate}
	\setcounter{enumi}{3}
		\item Given an open set $U$, $\lbrace V_i \rbrace$ an open covering of $U$ and $s \in \mathscr{F}(U)$ such that $s \restriction V_i = 0$ for all $i$, then one has $s = 0$.
		\item Given an open set $U$, $\lbrace V_i \rbrace$ an open covering of $U$ and elements $s_i \in \mathscr{F}(V_i)$ for each $i$ satisfying $s_i \restriction V_i \cap V_j = s_j \restriction V_i \cap V_j$ for all $i$ and $j$, then there exists an element $s \in \mathscr{F}(U)$ such that $s \restriction V_i = s_i$ for all $i$.  
	\end{enumerate}	
\end{definition}

%\begin{definition}[localization of a ring]

%\end{definition}

%\begin{definition}[spectrum of a ring]
	
%\end{definition}												

\begin{appendices}
	\addappheadtotoc
	
\chapter{Commutative Algebra}
\label{chap:commalg}

\section{Fields}
\label{sec:fields}
	
\end{appendices}						

		
\end{document}
