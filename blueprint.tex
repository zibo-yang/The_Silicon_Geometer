\documentclass[12pt]{article}

\usepackage[utf8]{inputenc}
\usepackage[T1]{fontenc}
\usepackage{babel}

\usepackage{amscd, amssymb, mathtools, amsthm, amsmath, mathrsfs}
\usepackage{enumerate}
% Palatino for main text and math
\usepackage[osf,sc]{mathpazo}

% Helvetica for sans serif
% (scaled to match size of Palatino)
\usepackage[scaled=0.90]{helvet}

% Bera Mono for monospaced
% (scaled to match size of Palatino)
\usepackage[scaled=0.85]{beramono}

\usepackage{hyperref}

% To write appendixes
\usepackage[page]{appendix}


\renewcommand{\thesubsection}{\arabic{subsection}}
%
%
\newenvironment{eq}{\begin{equation}}{\end{equation}}
%
\newenvironment{remarks}{{\bf Remarks}:\begin{enumerate}}{\end{enumerate}}
\newenvironment{examples}{{\bf Examples}:\begin{enumerate}}{\end{enumerate}}  
%
\newtheorem{proposition}{Proposition}[section]
\newtheorem{lemma}[proposition]{Lemma}
\newtheorem{definition}[proposition]{Definition}
\newtheorem{theorem}[proposition]{Theorem}
\newtheorem{cor}[proposition]{Corollary}
\newtheorem{conjecture}{Conjecture}
\newtheorem{pretheorem}[proposition]{Pretheorem}
\newtheorem{hypothesis}[proposition]{Hypothesis}
\newtheorem{example}[proposition]{Example}
\newtheorem*{rem}{Remark}
\newtheorem{remark}[proposition]{Remark}
\newtheorem{ex}[proposition]{Exercise}
\newtheorem{cond}[proposition]{Conditions}
\newtheorem{cons}[proposition]{Construction}
%
\newtheorem{problem}[proposition]{Problem}
\newtheorem{construction}[proposition]{Construction}
%
\newtheorem*{nota}{Notation}
\newtheorem{notation}{Notation}
\newtheorem*{axiom}{Axiom}

% For diagrams:
\usepackage{tikz-cd}
\usetikzlibrary{cd}

% To avoid automatic identation of paragraphs:
\usepackage[parfill]{parskip}

% To create an index:
\usepackage{makeidx}

% To make the index and bibliography appear in the table of contents:
\usepackage[nottoc]{tocbibind}

\makeindex

\begin{document}

\title{Blueprint}
\author{Anthony Bordg}
\date{\today}
\maketitle



\begin{definition}[zero divisor]
	Two elements $x$ and $y$ of a ring are zero divisors if $x \neq 0$, $y \neq 0$ and $x y = 0$.
\end{definition}

\begin{definition}[entire ring]
	A ring $R$ is said to be entire if it is commutative, $1 \neq 0$ and there are no zero divisors in $R$. 
\end{definition}

For the remainder of this text fix an entire ring $R$. 

\begin{definition}[ideal]
	An ideal $\mathfrak{a}$ of $R$ is a subset of $R$ which is an additive subgroup such that $R\mathfrak{a} = \mathfrak{a}$.
\end{definition}

\begin{ex}
	Prove that $\lbrace 0 \rbrace$ and $R$ are ideals of $R$.
\end{ex}

\begin{ex}
	Let $\mathfrak{a}$ and $\mathfrak{b}$ be two ideals of $R$. Check that $\mathfrak{a} \mathfrak{b}$ is also an ideal of $R$.
\end{ex}
		
\begin{definition}[prime ideal]
	A prime ideal $\mathfrak{p}$ of $R$ is an ideal $\mathfrak{p} \neq R$ such that $R/\mathfrak{p}$ is an entire ring. 	
\end{definition}

\begin{ex}
	Equivalently, a prime ideal is an ideal $\mathfrak{p} \neq R$ such that $x y \in \mathfrak{p}$ implies $x \in \mathfrak{p}$ or $y \in \mathfrak{p}$ for every elements $x$ and $y$ in $R$.
\end{ex}

\begin{remark}
	In a formal setting it is probably better to use the characterization above as the definition of prime ideals.
\end{remark}

\begin{remark}
	If $\mathfrak{p}$ is a prime ideal, then $1 \notin \mathfrak{p}$.
\end{remark}

\begin{ex}
	Let $\mathfrak{p}$ be a prime ideal and $S$ be the complement of $\mathfrak{p}$ in $R$. Prove $S$ is a multiplicative submonoid of $R$. 
\end{ex}		

\begin{notation}
	If $\mathfrak{a}$ is an ideal of $R$, $V(\mathfrak{a})$ denotes the set of all prime ideals of $R$ which contain $\mathfrak{a}$. 
\end{notation}

\begin{remark}
	On has $V(R) = \emptyset$ and $V(\lbrace 0 \rbrace)$ is the set of all prime ideals of $R$.
\end{remark}	

\begin{ex}
Let $\mathfrak{a}$ and $\mathfrak{b}$ be two ideals of $R$. One defines $\mathfrak{a} \mathfrak{b}$ to be the set \[
\lbrace a_1 b_1 + \dots + a_n b_n \mid a_i \in \mathfrak{a}, b_i \in \mathfrak{b}, n \in \mathbb{N} \rbrace.
\]
Check that $\mathfrak{a} \mathfrak{b}$ is an ideal of $R$.
\end{ex}

\begin{ex}
Prove the equality $V(\mathfrak{a} \mathfrak{b}) = V(\mathfrak{a}) \cup V(\mathfrak{b})$.	
\end{ex}

\begin{ex}
	Let $\lbrace \mathfrak{a}_i \rbrace$ be a family of ideals of $R$ indexed by a set $I$. One defines the set $\displaystyle \sum_{i \in I} \mathfrak{a}_i$ to be the set 
	\[
	\lbrace a_1 + \dots + a_n \mid n \in \mathbb{N}, a_i \in \mathfrak{a}_i \rbrace .
	\]
	Check that $\displaystyle \sum _{i \in I}\mathfrak{a}_i$ is an ideal of $R$.  
\end{ex}
	
\begin{ex}
	Prove the equality $V(\sum \mathfrak{a}_i) = \bigcap V(\mathfrak{a}_i)$ for any family of prime ideals.
\end{ex}

\begin{ex}
	Prove the subsets of the form $V(\mathfrak{a})$ as the closed subsets define a topology on the set of all primes ideals of $R$. 
\end{ex}

\begin{notation}
	Spec $R$ will denote the topological space of all prime ideals of $R$.
\end{notation}

\begin{definition}[presheaf of rings]
	Let $X$ be a topological space. A presheaf $\mathscr{F}$ of rings on $X$ consists of the following data:
	\begin{itemize}
		\item for every open set $U$, a ring $\mathscr{F}(U)$
		\item for every inclusion $V \subseteq U$ of open subsets, a morphism of rings $\rho_{UV}: \mathscr{F}(U) \rightarrow \mathscr{F}(V)$  
	\end{itemize}
satisfying 
	\begin{enumerate}
		\item $\mathscr{F}(\emptyset) = \lbrace 0 \rbrace$
		\item $\rho_{UU}$ is the identity map for every open subset $U$
		\item  If $W \subseteq V \subseteq U$ are three open subsets, then $\rho_{UW} = \rho_{VW} \circ \rho_{UV}$.
	\end{enumerate}
\end{definition}

\begin{notation}
	The elements of $\mathscr{F}(U)$ are sometimes called the sections of the presheaf $\mathscr{F}$ and given $s \in \mathscr{F}(U)$, $s\restriction V$ denotes the element $\rho_{UV}(s)$.
\end{notation}

\begin{definition}[morphism of presheaves of rings]
	A morphism $\phi: \mathscr{F} \rightarrow \mathscr{F}'$ of sheaves of rings on a topological space $X$ is given by a morphism $\phi_U: \mathscr{F}(U) \rightarrow \mathscr{F}'(U)$ for each open subset $U$ of $X$ such that  the following diagram commutes
	\[
	\begin{tikzcd}
	\mathscr{F}(U) \arrow[r, "\phi_U"] \arrow[d, "\rho_{UV}"] & \mathscr{F}'(U) \arrow[d, "\rho'_{UV}"] \\
	\mathscr{F}(V) \arrow[r, "\phi_V"] & \mathscr{F}'(V) 
	\end{tikzcd}
	\]
	for every inclusion $V \subseteq U$. \\
	An isomorphism of presheaves is a morphism which has a two-sided inverse.
\end{definition}		

\begin{definition}[sheaf of rings]
	A sheaf of rings on a topological space $X$ is a presheaf on $X$ that satisfies the following additional properties:
	\begin{enumerate}
	\setcounter{enumi}{3}
		\item Given an open set $U$, $\lbrace V_i \rbrace$ an open covering of $U$ and $s \in \mathscr{F}(U)$ such that $s \restriction V_i = 0$ for all $i$, then one has $s = 0$.
		\item Given an open set $U$, $\lbrace V_i \rbrace$ an open covering of $U$ and elements $s_i \in \mathscr{F}(V_i)$ for each $i$ satisfying $s_i \restriction V_i \cap V_j = s_j \restriction V_i \cap V_j$ for all $i$ and $j$, then there exists an element $s \in \mathscr{F}(U)$ such that $s \restriction V_i = s_i$ for all $i$.  
	\end{enumerate}	
\end{definition}

\begin{definition}[morphism of sheaves of rings]
	A morphism of sheaves of rings is a morphism of presheaves of rings.
\end{definition}

\begin{ex}
	Let $X$ be a topological space, $\mathscr{F}$ a sheaf of rings on $X$ and $U$ an open subset of $X$. \\ Prove that $\mathscr{F}|_U$, defined by $\mathscr{F}|_U(V) \coloneqq \mathscr{F}(U \cap V)$, is a sheaf of rings on $U$ for the induced topology.
\end{ex}		

\begin{cons}
	Let $f: X \rightarrow Y$ be a continuous map between topological spaces and $\mathscr{F}$ a sheaf of rings on $X$. Given an open subset $V$ of $Y$, define $(f_{*} \mathscr{F})(V)$ to be $\mathscr{F} (f^{-1}(Y))$.  
\end{cons}

\begin{ex}
	Prove $f_{*} \mathscr{F}$ above is a sheaf of rings on the topological space $Y$.  
\end{ex}

\begin{definition}[direct image sheaf]
	Let $f: X \rightarrow Y$ be a continuous map between topological spaces and $\mathscr{F}$ a sheaf of rings on $X$. The sheaf $f_{*} \mathscr{F}$ is the direct image of $\mathscr{F}$. 
\end{definition}		

\begin{cons}
	Let $S$ be a multiplicative submonoid of $R$. We consider pairs $(r, s)$ with $r \in R$ and $s \in S$ and we define the following relation on them
	\[
	(r, s) \sim (r', s')
	\]
	if and only if there exists $s_1 \in S$ such that $s_1(s' r - s r') = 0$. One checks that the relation $\sim$ is an equivalence relation. The equivalence class of a pair $(r, s)$ is denoted by $r/s$ and the set of equivalence classes is denoted by $S^{-1} R$. \\
	We define the following addition on $S^{-1} R$.
	\[
	\frac{r}{s} + \frac{r'}{s'} = \frac{r s' + r' s}{s s'}
	\]
	We also define the following multiplication on $S^{-1} R$. 
	\[
	\frac{r}{s} \times \frac{r'}{s'} = \frac{r r'}{s s'}
	\]
\end{cons}

\begin{ex}
	Prove $(S^{-1} R, 0/1, +, 1/1, \times)$ is a ring.
\end{ex}		

\begin{definition}[quotient ring]
	Let $S$ be a multiplicative submonoid of $R$. The quotient ring of $R$ by $S$ is $S^{-1} R$. 
\end{definition}

\begin{definition}[local ring at a prime ideal] \label{localringat}
	Let $\mathfrak{p}$ be a prime ideal of $R$ and $S$ the complement of $\mathfrak{p}$ in $R$. The local ring of $R$ at $\mathfrak{p}$, denoted $R_{\mathfrak{p}}$, is $S^{-1} R$. 
\end{definition}	

\begin{cons}
	Let $U$ be an open subset of Spec $R$. We define $\mathscr{O}_R(U)$ to be the set of all functions 
	\[
	s: U \rightarrow \bigcup \limits_{\mathfrak{p} \in U} R_{\mathfrak{p}}
	\]
	such that $s(\mathfrak{p}) \in R_{\mathfrak{p}}$ for every $\mathfrak{p} \in U$ and such that for every $\mathfrak{p} \in U$, there exist a neighborhood $V$ of $\mathfrak{p}$, contained in $U$, and elements $r, f \in R$, such that for each $\mathfrak{q} \in V$, $f \notin \mathfrak{q}$ and $s(\mathfrak{q}) = r/f$. \\
	We endow $\mathscr{O}_R(U)$ with the sum and product of such functions and we take for the multiplicative unit the function that maps each $\mathfrak{p}$ on the multiplicative unit of $R_{\mathfrak{p}}$. \\
	Last, if $V \subseteq U$, then the morphism of rings $\mathscr{O}_R(U) \rightarrow \mathscr{O}_R(V)$ maps each $s \in \mathscr{O}_R(U)$ on its restriction to $V$. 
\end{cons}

\begin{ex}
	Prove $\mathscr{O}_R$ is a sheaf of rings on Spec $R$. 
\end{ex}			

\begin{definition}[spectrum of a ring]
	The spectrum of the ring $R$ is the pair consisting of the topological space $Spec R$ together with the sheaf of rings $\mathscr{O}_R$.
\end{definition}

\begin{definition}[ringed space]
	A ringed space is a pair $(X, \mathscr{O}(X))$, where $X$ is a topological space and $\mathscr{O}(X)$ is a sheaf of rings on $X$.
\end{definition}

\begin{definition}[morphism of ringed spaces]
	A morphism of ringed spaces from $(X, \mathscr{O}(X))$ to $(Y, \mathscr{O}(Y))$ is a pair $(f, \phi_f)$ consisting of a continuous map $f: X \rightarrow Y$ between topological spaces and a morphism $\phi_f: \mathscr{O}_Y \rightarrow f_{*} \mathscr{O}_X$ of sheaves of rings.   
\end{definition}

\begin{construction}
	Let $X$ be a topological space and $\mathscr{F}$ a sheaf of rings on $X$. The set $I$ will denote a subset of the set of all open subsets of $X$. \\
	Given $U, V \in I$, $s \in \mathscr{F}(U)$ and $t \in \mathscr{F}(V)$,  we says that 
	$s \sim t$ if and only if there exists $W \in I$ such that $W \subseteq U \cap V$ and $s \restriction W = t \restriction W$. One checks that $\sim$ is an equivalence relation. \\
	Now, we consider the quotient of the disjoint union of the $\mathscr{F}(U)$'s.
	\[
	\varinjlim_I \mathscr{F}(U) \coloneqq \coprod_{U \in I} \mathscr{F}(U) \bigg/ \sim 
	\]
	Note that for any two elements $U, V \in I$, one has $[0_U] = [0_V]$ and $[1_U] = [1_V]$, where $[0_U]$ (\textit{resp.} $[1_U]$) denotes the class of the zero element (\textit{resp.} the multiplicative unit) of the ring $\mathscr{F}(U)$. \\
	Last, on each component $\mathscr{F}(U)$ the ring operations of $\displaystyle \varinjlim_{I} \mathscr{F}(U)$ are those of $\mathscr{F}(U)$. 
\end{construction}

\begin{ex}
	Prove $(\displaystyle \varinjlim_I \mathscr{F}(U), 0, +, 1, \times)$ is a ring. 
\end{ex}

\begin{definition}[direct limit]
	Let $X$ be a topological space, $\mathscr{F}$ a sheaf of rings on $X$ and $I$ a set of open subsets of $X$. The direct limit over $I$ is the ring $\displaystyle \varinjlim_I \mathscr{F}(U)$ or simply $\varinjlim \mathscr{F}(U)$.
\end{definition}

\begin{definition}[stalks of a presheaf]
	Let $X$ be a topological space, $\mathscr{F}$ a sheaf of rings on $X$ and $x$ an element of $X$. \\
	The stalk $\mathscr{F}_x$ of $\mathscr{F}$ at $x$ is the direct limit over the set of all the open neighborhoods of $x$.   
\end{definition}

\begin{definition}[maximal ideal]
	Let $\mathfrak{m}$ be an ideal of $R$. The ideal $\mathfrak{m}$ is maximal if $\mathfrak{m} \neq R$ and there is no ideal $\mathfrak{a} \neq R$ such that $\mathfrak{m} \subset \mathfrak{a}$.
\end{definition}

\begin{definition}[local ring]
	A ring is a local ring if it is commutative and it has a unique maximal ideal. 	
\end{definition}

\begin{ex}
	Prove $R_{\mathfrak{p}}$ \ref{localringat} is a local ring.
\end{ex}

\begin{definition}[local homomorphism]
	Let $A$ and $B$ be two local rings and $\mathfrak{m}_A$, $\mathfrak{m}_B$ their respective maximal ideals. \\
	A local homomorphism $f: A \rightarrow B$ is a morphism of rings such that $f^{-1} (\mathfrak{m}_B) = \mathfrak{m}_A$. 
\end{definition}		

\begin{definition}[locally ringed space]
	A locally ringed space is a ringed space $(X, \mathscr{O}_X)$ such that the stalk $(\mathscr{O}_X)_x$ is a local ring for every $x \in X$.
\end{definition}

\begin{ex}
	Prove $(Spec R, \mathscr{O}_R)$ is a locally ringed space.
\end{ex}

\begin{cons}
	Let $(X, \mathscr{O}_X)$, $(Y, \mathscr{O}_Y)$ be two ringed spaces and $(f, \phi_f)$ a morphism between them. Given $x \in X$, consider 
	\[
	I \coloneqq \lbrace V \mid V \, \text{is an open neighborhhood of}\, f(x) \rbrace .
	\]
	Since we have a morphism $\phi_f: \mathscr{O}_Y \rightarrow f_* \mathscr{O}_X $ of sheaves on $Y$, we get a map between the direct limits over $I$. 
	\[
	(\mathscr{O}_Y)_{f(x)} \rightarrow \displaystyle \varinjlim_I \mathscr{O}_X (f^{-1} V)
	\]
	Last, there is a natural inclusion from $\displaystyle \varinjlim_I \mathscr{O}_X (f^{-1} V)$ to $(\mathscr{O}_X)_x$, so eventually we get an induced morphism of rings from $(\mathscr{O}_Y)_{f(x)}$ to $(\mathscr{O}_X)_x$ which we will denote $\phi_{f, x}$. 
\end{cons}	
				
\begin{definition}[morphism of locally ringed spaces]
	A morphism of locally ringed spaces from $(X, \mathscr{O}_X)$ to $(Y, \mathscr{O}_Y)$ is a morphism $(f, \phi_f)$ between ringed spaces such that the induced map of local rings $\phi_{f, x}: (\mathscr{O}_Y)_{f(x)} \rightarrow (\mathscr{O}_X)_x$ is a local homomorphism for every $x \in X$.  
\end{definition}	

\begin{definition}[affine scheme]
	An affine scheme is a locally ringed space $(X, \mathscr{O}_X)$ which is isomorphic (as a locally ringed space) to the spectrum $(Spec R, \mathscr{O}_R)$ for some ring $R$. 
\end{definition}

\begin{definition}[scheme]
	A scheme is a locally ringed space $(X, \mathscr{O}_X)$ in which every point has an open neighborhood $U$ such that the topological space $U$, together with the sheaf $\mathscr{O}_X | _U$, is an affine scheme.
\end{definition}

\begin{appendices}
	\addappheadtotoc
	
\section{Solutions of exercises}
	
	% To be completed
	
\end{appendices}				

		
\end{document}
