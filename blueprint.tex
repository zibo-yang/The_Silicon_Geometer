\documentclass[12pt]{report}

\usepackage[utf8]{inputenc}
\usepackage[T1]{fontenc}
\usepackage{babel}

\usepackage{amscd, amssymb, mathtools, amsthm, amsmath}
\usepackage{enumerate}
% Palatino for main text and math
\usepackage[osf,sc]{mathpazo}

% Helvetica for sans serif
% (scaled to match size of Palatino)
\usepackage[scaled=0.90]{helvet}

% Bera Mono for monospaced
% (scaled to match size of Palatino)
\usepackage[scaled=0.85]{beramono}

\usepackage{hyperref}

% To write appendixes
\usepackage[page]{appendix}


\renewcommand{\thesubsection}{\arabic{subsection}}
%
%
\newenvironment{eq}{\begin{equation}}{\end{equation}}
%
\newenvironment{remarks}{{\bf Remarks}:\begin{enumerate}}{\end{enumerate}}
\newenvironment{examples}{{\bf Examples}:\begin{enumerate}}{\end{enumerate}}  
%
\newtheorem{proposition}{Proposition}[section]
\newtheorem{lemma}[proposition]{Lemma}
\newtheorem{definition}[proposition]{Definition}
\newtheorem{theorem}[proposition]{Theorem}
\newtheorem{cor}[proposition]{Corollary}
\newtheorem{conjecture}{Conjecture}
\newtheorem{pretheorem}[proposition]{Pretheorem}
\newtheorem{hypothesis}[proposition]{Hypothesis}
\newtheorem{example}[proposition]{Example}
\newtheorem*{rem}{Remark}
\newtheorem{remark}[proposition]{Remark}
\newtheorem{ex}[proposition]{Exercise}
\newtheorem{cond}[proposition]{Conditions}
\newtheorem{cons}[proposition]{Construction}
%
\newtheorem{problem}[proposition]{Problem}
\newtheorem{construction}[proposition]{Construction}
%
\newtheorem*{nota}{Notation}
\newtheorem{notation}{Notation}
\newtheorem*{axiom}{Axiom}

% For diagrams:
\usepackage{tikz-cd}
\usetikzlibrary{cd}

% To avoid automatic identation of paragraphs:
\usepackage[parfill]{parskip}

% To create an index:
\usepackage{makeidx}

% To make the index and bibliography appear in the table of contents:
\usepackage[nottoc]{tocbibind}

\makeindex

\begin{document}
	
\title{Blueprint}
\author{}
\date{\today}
\maketitle


Let $k$ be an algebraically closed field and $A$ denote the ring $k \left[ x_1, \dots, x_n \right]$. \\

\begin{definition}[affine $n$-space] 
	The affine $n$-space over $k$, denoted $\mathbb{A}^n_k$ or simply $\mathbb{A}^n$, is the set of all $n$-tuples of elements of $k$.  
\end{definition}

\begin{remark}
	An element $P$ of $\mathbb{A}^n$ will be called a point. \\
	If $P = (a_1, \dots, a_n)$ with $a_i \in k$, then $a_i$ will be called the $i$th coordinate of $P$.
\end{remark}

\begin{notation}
	if $P = (a_1, \dots, a_n)$ (\textit{resp.} $f$) belongs to $\mathbb{A}^n$ (\textit{resp.} $A$), then $f(P)$ is $f(a_1, \dots, a_n)$.
\end{notation}		

\begin{definition}[zero set]
	If $T$ is any subset of $A$, one defines the zero set of $T$, denoted $Z(T)$, to be the common zeros of all the elements of $T$. \\
	In other words one has
	\[
	Z(T) \coloneqq \lbrace P \in \mathbb{A}^n \mid f(P) = 0 \quad \text{for all}\, f \in T \rbrace \, . 
	\]
\end{definition}

\begin{definition}[algebraic set]
	A subset $Y$ of $\mathbb{A}^n$ is an algebraic set if there exists a subset $T \subseteq A$ such that $Y = Z(T)$. 
\end{definition}

\begin{ex}
	Prove the empty space $\emptyset$ and the whole space $\mathbb{A}^n$ are algebraic sets.
\end{ex}

\begin{ex}
	Prove the intersection of any family of algebraic sets is an algebraic set.
\end{ex}

\begin{ex}
	Prove the union of two algebraic sets is an algebraic set.
\end{ex}		

\begin{definition}[Zariski topology]
	One defines the Zariski topology on $\mathbb{A}^n$ by taking the open sets to be the complements of the algebraic sets.
\end{definition}

\begin{ex}
	Prove the Zariski topology is a topology. 
\end{ex}

\begin{definition}[irreducible subset]
	A nonempty subset $Y$ of a topological space $X$ is irreducible if $Y$ cannot be expressed as the union of two proper closed subsets of $X$. 
\end{definition}

\begin{definition}[affine algebraic variety]
	An affine algebraic variety or simply an affine variety is an irreducible closed subset of $\mathbb{A}^n$ endowed with the induced topology. 
\end{definition}

\begin{definition}[quasi-affine variety]
	A quasi-affine variety is an open subset of an affine variety. 
\end{definition}					

		
\end{document}
